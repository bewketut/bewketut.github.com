%% License: BSD style (Berkley) (i.e. Put the Copyright owner's name always)
%% Writer and Copyright (to): Bewketu(Bilal) Tadilo (2016-17)
\documentclass[11pt,a4paper,oneside]{l3doc}%, ,fleqn oneside]
\usepackage[bottom=2cm,top=2cm,right=2cm,text={7in,10in},centering]{geometry}
\usepackage[no-math]{fontspec}
\usepackage{graphicx}
\usepackage{booktabs}
\usepackage{longtable}
\usepackage{polyglossia}
\usepackage{ifxetex}
\usepackage{xcolor} % Required for specifying colors by name
\usepackage{expl3}
\usepackage{hyperref}
\usepackage{fancyhdr}
\usepackage{fancybox}
\usepackage{color}
\usepackage{calc}
\definecolor{shadecolor}{rgb}{1, 0, 0}
\pagestyle{fancy}
\fancyhf{}
\renewcommand{\headrulewidth}{0pt}
\renewcommand{\footrulewidth}{0.6pt}
\renewcommand{\footrule}{{\color{black}%
\vskip-\footruleskip\vskip-\footrulewidth
\hrule width\textwidth  height\footrulewidth\vskip\footruleskip}}

\fancypagestyle{plain}{
\fancyhf{}
\noindent
\fancyhead[LO,RE]{
\noindent
\thisfancyput(3.1in,-4.5in){%
\setlength{\unitlength}{1in}\fancyoval(7.7,9.9)}
}
\fancyfoot[LO]{\vskip-1.1cm\hskip-0.6cm\large\thepage}
\fancyfoot[RE]{\large\thepage\vskip-1.1cm\hskip-0.6cm}
}


\newfontfamily\arabicfont[Script=Arabic, Scale = 1.5]{mequran}%KFGQPC Uthmanic Script HAFS}% Taha Naskh}
%\newfontfamily\arabicfonttt[Script = Arabic, Scale = 1.5]{Amiri}%KFGQPC Uthmanic Script HAFS}% Taha Naskh}
\newfontfamily\amharicfont[Script=Ethiopic, Scale = 1.1]{Abyssinica SIL}
\setmainlanguage{arabic}
\setotherlanguage{amharic}
\selectbackgroundlanguage{arabic}
\newcommand{\rqt}{\rq\rq{}}
\newcommand{\enqt}[1]{\rq\rq{}#1\rq\rq{}}
\newcommand{\textamh}[1]{\noindent\raggedright\LR{\noindent\begin{amharic}#1\end{amharic}\noindent}}
\begin{document}
\vspace*{3cm}
\centering {\Large \textamharic{መጽሐፈ ሄኖክ}}\\ \\
 {\Large كتب الهينوك }\\

{\Large \textamharic{\today -}}
\clearpage
\begin{longtable}{
@{}
p{.5\textwidth}
@{~~~~~~~~~~~~~}
p{0.5\textwidth}
@{}
}
\textamh{{\Large 1}\ \    ቃለ ፡ በረከት ፡ ዘሄኖክ ፡ ዘከመ ፡ ባረከ ፡ ኅሩያነ ፡ ወጻድቃነ ፡ እለ ፡ ሀለዉ ፡ ይኩኑ ፡ በዕለተ ፡ ምንዳቤ ፡ ለአሰስሎ ፡ ኵሉ ፡ እኩያን ፡ ወረሲዓን ። ወአውሥአ ፡ ወይቤ ፡ ሄኖክ ፡ ብእሲ ፡ ጻድቅ ፡ ዘእምኀበ ፡ እግዚአብሔር ፡ እንዘ ፡ አዕይንቲሁ ፡ ክሡታት ፡ ወይሬኢ ፡ ራእየ ፡ ቅዱሰ ፡ ዘበሰማያት ፡ ዘአርአዩኒ ፡ መላእክት ፡ ወሰማዕኩ ፡ እምኀቤሆሙ ፡ ኵሎ ፡ ወአእመርኩ ፡ አነ ፡ ዘእሬኢ ፡ ወአኮ ፡ ለዝ ፡ ትውልድ ፡ አላ ፡ ለዘይመጽእ ፡ ትውልድ ፡ ርኁቃን ። በእንተ ፡ ኅሩያን ፡ እቤ ፡ ወአውሣእኩ ፡ በእንቲአሆሙ ፡ ምስለ ፡ ዘይወፅእ ፡ ቅዱስ ፡ ወዐቢይ ፡ እማኅደሩ ፡ ወአምላከ ፡ ዓለም ። ወእምህየ ፡ ይከይድ ፡ ዲበ ፡ ሲና ፡ ደብር ፡ ወያስተርኢ ፡ በትዕይንቱ ፡ ወያስተርኢ ፡ በጽንዐ ፡ ኀይሉ ፡ እምሰማይ ። ወይፈርህ ፡ ኵሉ ፡ ወያድለቀልቁ ፡ ትጉሃን ፡ ወይነሥኦሙ ፡ ፍርሀት ፡ ወረዓድ ፡ ዐቢይ ፡ እስከ ፡ አጽናፈ ፡ ምድር ። ወይደነግፁ ፡ አድባር ፡ ነዋኃን ፡ ወይቴሐቱ ፡ አውግር ፡ ነዋኃት ፡ ወይትመሰዉ ፡ ከመ ፡ መዓረ ፡ ግራ ፡ እምላህብ ። ወትሠጠም ፡ ምድር ፡ ወኵሉ ፡ ዘውስተ ፡ ምድር ፡ ይትሐጐል ፡ ወይከውን ፡ ፍትሕ ፡ ላዕለ ፡ ኵሉ ፡ ወላዕለ ፡ ጻድቃን ፡ ኵሎሙ ። ለጻድቃንሰ ፡ ሰላመ ፡ ይገብር ፡ ሎሙ ፡ ወየዐቅቦሙ ፡ ለኅሩያን ፡ ወይከውን ፡ ሣህል ፡ ላዕሌሆሙ ፡ ወይከውኑ ፡ ኵሎሙ ፡ ዘአምላክ ፡ ወይሤርሑ ፡ ወይትባረኩ ፡ ወይሀርህ ፡ ሎሙ ፡ ብርሃነ ፡ አምላክ ። ወናሁ ፡ መጽአ ፡ በትእልፈት ፡ ቅዱሳን ፡ ከመ ፡ ይግበር ፡ ፍትሐ ፡ ላዕሌሆሙ ፡ ወያሐጕሎሙ ፡ ለረሲዓን ፡ ወይትዋቀስ ፡ ኵሎ ፡ ዘሥጋ ፡ በእንተ ፡ ኵሉ ፡ ዘገብሩ ፡ ወረሰዩ ፡ ላዕሌሁ ፡ ኃጥአን ፡ ወረሲዓን ።} & قول البركة الخينك كمابركاصلحين وصدقين \\
\textamh{{\Large 2}\ \    ጠየቁ ፡ ኵሎ ፡ ዘውስተ ፡ ሰማይ ፡ ግብረ ፡ እፎ ፡ ኢይመይጡ ፡ ፍናዊሆሙ ፡ ብርሃናት ፡ ዘውስተ ፡ ሰማይ ፡ ከመ ፡ ኵሉ ፡ ይሠርቅ ፡ ወየዐርብ ፡ ሥሩዕ ፡ ኵሉ ፡ በበዘመኑ ፡ ወኢይትዐደዉ ፡ እምትእዛዞሙ ። ርእይዋ ፡ ለምድር ፡ ወለብዉ ፡ በእንተ ፡ ምግባር ፡ ዘይትገበር ፡ ላዕሌሃ ፡ እምቀዳሚ ፡ እስከ ፡ ተፍጻሜቱ ፡ ከመ ፡ እይትመየጥ ፡ ኵሉ ፡ ግብሩ ፡ ለአምላክ ፡ እንዘ ፡ ያስተርኢ ። ርእይዎ ፡ ለሐጋይ ፡ ወለክረምት ፡ ከመ ፡ ኵላ ፡ ምድር ፡ መልአት ፡ ማየ ፡ ወደመና ፡ ወጠል ፡ ወዝናም ፡ የዐርፍ ፡ ላዕሌሃ ።} & \\
\textamh{{\Large 3}\ \   ጠየቁ ፡ ወርኢኩ ፡ ከመ ፡ ኵሉ ፡ ዐፀው ፡ እፎ ፡ ያስተርእዩ ፡ ከመ ፡ ይቡስ ፡ ወኵሉ ፡ አቍጽሊሆሙ ፡ ንጉፋት ፡ ዘእንበለ ፡ ፲ወ፬ዐፀው ፡ ዘኢይትነገፍኡ ፡ እለ ፡ ይጸንሑ ፡ እምብሉይ ፡ እስከ ፡ ይመጽእ ፡ ሐዲስ ፡ እም፪ወእም፫ክረምት ።} & \\
\textamh{{\Large 4}\ \   ወዳግመ ፡ ጠየቁ ፡ መዋዕለ ፡ ሐጋይ ፡ ከመ ፡ ኮነ ፡ ፀሐይ ፡ ላእሌሃ ፡ በቅድሜሃ ፡ ወአንትሙሰ ፡ ተኀሥሡ ፡ ምጽላለ ፡ ወጽላሎተ ፡ በእንተ ፡ ዋዕየ ፡ ፀሐይ ፡ ወምድርኒ ፡ ትውዒ ፡ እሙቀተ ፡ ሐሩር ፡ ወአንትሙሰ ፡ ኢትክሉ ፡ ከይዶታ ፡ ለምድር ፡ ወኢኰኵሐ ፡ በእንተ ፡ ዋእያ ።} & \\
\textamh{{\Large 5}\ \    ጠየቁ ፡ እፎ ፡ ዕፀው ፡ በሐመልማለ ፡ አቍጽል ፡ ይትክደኑ ፡ ወይፈርዩ ፡ ወለብዉ ፡ በእንተ ፡ ኵሉ ፡ ወአእምሩ ፡ በከመ ፡ ገብረ ፡ ለክሙ ፡ እሎንተ ፡ ኵሎሙ ፡ ዘሕያው ፡ ለዓለም ። ወምግባሩ ፡ ቅድሜሁ ፡ ለለዓመት ፡ ዘይከውን ፡ ወኵሉ ፡ ምግባሩ ፡ ይትቀነዩ ፡ ሎቱ ፡ ወኢይትመየጡ ፡ አላ ፡ በከመ ፡ ሠርዐ ፡ አምላክ ፡ ከመዝ ፡ ይትገበር ፡ ኵሉ ። ወርእዩ ፡ እፎ ፡ አብህርት ፡ ወአፍላግ ፡ ኅቡረ ፡ ይፌጽሙ ፡ ግብሮሙ ። ወአንትሙሰ ፡ ኢተዐገሥክሙ ፡ ወኢገበርክሙ ፡ ትእዛዘ ፡ እግዚእ ፡ እላ ፡ ተዐደውክሙ ፡ ወሐመይክሙ ፡ ዓቢያተ ፡ ወድሩካተ ፡ ቃላተ ፡ በአፍ ፡ ርኩስት ፡ ዘዚአክሙ ፡ ላዐለ ፡ ዕበየ ፡ ዚአሁ ፤ ይቡሳነ ፡ ልብ ፡ ኢትከውነክሙ ፡ ሰላም ። ወበእንተዝ ፡ አንትሙ ፡ መዋዕሊክሙ ፡ ትረግሙ ፡ ወዓመታተ ፡ ሕይወትክሙ ፡ ተሐጕሉ ፡ ወይበዝኅ ፡ መርገም ፡ ዘለዓለም ፡ ወኢይከውነክሙ ፡ ሣህል ። በውእቱ ፡ መዋዕል ፡ ትሁቡ ፡ ሰላመ ፡ ዚአክሙ ፡ በርግመት ፡ ዘለዓለም ፡ ለኵሉ ፡ ጻድቃን ፡ ወኪያክሙ ፡ ይረግሙ ፡ ኃጥኣን ፡ ዘልፈ ፡ ወለክሙ ፡ ኅቡረ ፡ ምስለ ፡ ኃጥኣን ። ወለኅሩያንሰ ፡ ይከውን ፡ ብርሃን ፡ ወፍሥሓ ፡ ወሰላም ፡ ወእሙንቱ ፡ ይወርስዋ ፡ ለምድር ፡ ወለክሙሰ ፡ ረሲዓን ፡ ይከውነክሙ ፡ ርግመት ። ወአመሂ ፡ ይትወሀቦሙ ፡ ለኅሩያን ፡ ጥበብ ፡ ወኵሎሙ ፡ እሎንቱ ፡ የሐይዉ ፡ ወኢይደግሙ ፡ አበሳ ፡ ኢበረሲዕ ፡ ወኢበትዕቢት ፡ አላ ፡ ይገንዩ ፡ ዘቦሙ ፡ ጥበብ ፡ ኢይደግሙ ፡ አብሶ ። ወኢይትኴነኑ ፡ ኵሎ ፡ መዋእለ ፡ ሕይወቶሙ ፡ ወኢይመውቱ ፡ በመቅሠፍት ፡ ወኢበመዓት ፡ አላ ፡ ኍልቈ ፡ መዋዕለ ፡ ሕይወቶሙ ፡ ይፌጽሙ ፡ ወይልህቅ ፡ ሕይወቶሙ ፡ በሰላም ፡ ወዓመታተ ፡ ፍሥሓሆሙ ፡ ይበዝኅ ፡ በሐሤት ፡ ወበሰላም ፡ ዘለዓለም ፡ ውስተ ፡ ኵሉ ፡ መዋእለ ፡ ሕይወቶሙ ።} & \\
\textamh{{\Large 6}\ \    ወኮነ ፡ እምዘ ፡ በዝኁ ፡ ውሉደ ፡ ሰብእ ፡ በእማንቱ ፡ መዋእል ፡ ተወልደ ፡ ሎሙ ፡ አዋልድ ፡ ሠናያት ፡ ወላህያት ። ወርእዩ ፡ ኪያሆን ፡ መላእክት ፡ ውሉደ ፡ ሰማያት ፡ ወፈተውዎን ፡ ወይቤሉ ፡ በበይናቲሆሙ ፡ ንዑ ፡ ንኅረይ ፡ ለነ ፡ አንስተ ፡ እምውሉደ ፡ ሰብእ ፡ ወንለድ ፡ ለነ ፡ ውሉደ ። ወይቤሎሙ ፡ ስምያዛ ፡ ዘውእቱ ፡ መልአኮሙ ፡ እፈርህ ፡ ዮጊ ፡ ኢትፈቅዱ ፡ ይትገበር ፡ ዝንቱ ፡ ግብር ፡ ወእከውን ፡ አነ ፡ ባሕቲትየ ፡ ፈዳዪሃ ፡ ለዛቲ ፡ ኅጢአት ፡ ዐባይ ። ወአውሥኡ ፡ ሎቱ ፡ ኵሎሙ ፡ ወይቤሉ ፡ መሐላ ፡ ንምሐል ፡ ኵልነ ፡ ወንትዋገዝ ፡ በበይናቲነ ፡ ከመ ፡ ኢንሚጣ ፡ ለዛቲ ፡ ምክር ፡ ወንግበራ ፡ ለዛቲ ፡ ምክር ፡ ግብረ ። አሜሃ ፡ መሐሉ ፡ ኵሎሙ ፡ ኅቡረ ፡ ወአውገዙ ፡ ኵሎሙ ፡ በበይናቲሆሙ ፡ ቦቱ ፡ ወኮኑ ፡ ኵሎሙ ፡ ፪፻ ። ወወረዱ ፡ ውስተ ፡ አርዲስ ፡ ዝውእቱ ፡ ድማሑ ፡ ለደብረ ፡ አርሞን ፡ ወጸውዕዎ ፡ ለደብረ ፡ አርሞን ፡ እስመ ፡ መሐሉ ፡ ቦቱ ፡ ወአውገዙ ፡ በበይናቲሆሙ ። ወዝንቱ ፡ አስማቲሆሙ ፡ ለመላእክቲሆሙ ፡ ስምያዛ ፡ ዘውእቱ ፡ መልአኮሙ ፡ ኡራኪበራሜኤል ፡ አኪበኤል ፡ ጣሚኤል ፡ ራሙኤል ፡ ዳንኤል ፡ ኤዜቄኤል ፡ ሰራቁያል ፡ አሳኤል ፡ አርምርስ ፡ በጥረአል ፡ አናንኢ ፡ ዘቄቤ ፡ ሰምሳዌኤል ፡ ሰርተኤል ፡ ጡርኤል ፡ ዮምያኤል ፡ አራዝያል ። እሉ ፡ እሙንቱ ፡ ሀበይቶሙ ፡ ለ፪፻መላእክት ፡ ወባዕዳን ፡ ኵሉ ፡ ምስሌሆሙ ።} & \\
\textamh{{\Large 7}\ \    ወነሥኡ ፡ ሎሙ ፡ አንስትያ ፡ ወኀረየ ፡ ኵሉ ፡ ለለርእሱ ፡ አሐተ ፡ አሐተ ፡ ወወጠኑ ፡ ይባኡ ፡ ኀቤሆን ፡ ወተደመሩ ፡ ምስሌሆን ፡ ወመሐርዎን ፡ ሥራያተ ፡ ወስብዓታተ ፡ ወመቲረ ፡ ሥርው ፡ ወዕፀው ፡ አመርዎን ። ወእማንቱሰ ፡ ፀንሳ ፡ ወወለዳ ፡ ረዓይተ ፡ ዐበይተ ፡ ወቆሞሙ ፡ በበ ፡ ፴፻ ፡ በእመት ። እሉ ፡ በልዑ ፡ ኵሎ ፡ ፃማ ፡ ሰብእ ፡ እስከ ፡ ስእንዎሙ ፡ ሴስዮተ ፡ ሰብእ ። ወተመይጡ ፡ ረዓይት ፡ ላዕሌሆሙ ፡ ይብልዕዎሙ ፡ ለሰብእ ። ወወጠኑ ፡ የአብሱ ፡ በአዕዋፍ ፡ ወዲበ ፡ አራዊት ፡ ወበዘይትሐወስ ፡ ወበዓሣት ፡ ወሥጋሆሙ ፡ በበይናቲሆሙ ፡ ይትባልዑ ፡ ወደመ ፡ ይስትዩ ፡ እምኔሃ ። አሜሃ ፡ ምድር ፡ ሰከየቶሙ ፡ ለዐማፅያን ።} & \\
\textamh{{\Large 8}\ \    ወአዛዝኤል ፡ መሐሮሙ ፡ ለሰብእ ፡ ገቢረ ፡ አስይፍት ፡ ወመጥባሕት ፡ ወወልታ ፡ ወድርዓ ፡ እንግድዓ ፡ ወአርአዮሙ ፡ ዘእምድኅሬሆሙ ፡ ወምግባረሆሙ ፡ አውቃፋተ ፡ ወሠርጐ ፡ ወተኵሕሎተ ፡ ወአሠንዮ ፡ ቀራንብት ፡ ወእብነ ፡ እምኵሉ ፡ እብን ፡ ክቡረ ፡ ወኅሩየ ፡ ወኵሎ ፡ ጥምዓታተ ፡ ኅብር ፡ ወተውላጠ ፡ ዓለም ። ወኮነ ፡ ርስዐት ፡ ዓቢይ ፡ ወብዙኅ ፡ ዘምዎ ፤ ወስሕቱ ፡ ወማሰና ፡ ኵሉ ፡ ፍናዊሆሙ ። አሜዛራክ ፡ መሀረ ፡ ኵሎ ፡ መሳብዕያነ ፡ ወመታርያነ ፡ ሥርዋት ፤ አርማሮስ ፡ ፈትሐ ፡ ስብዓታተ ፤ ወበረቅዓል ፡ ረዓይያነ ፡ ከዋክብት ፤ ወኮከብኤል ፡ ትእምርታተ ፤ ወጥምኤል ፡ መሀረ ፡ ራእየ ፡ ኮከብ ፤ ወአስራድኤል ፡ መሀረ ፡ ሩፀተ ፡ ወርኅ ። ወበኅጕለተ ፡ ሰብእ ፡ ጸርሑ ፡ ወበጽሐ ፡ ቃሎሙ ፡ ሰማየ ።} & \\
\textamh{{\Large 9}\ \  ወአሜሃ ፡ ሐወጹ ፡ ሚካኤል ፡ ወገብርኤል ፡ ወሱርያን ፡ ወኡርያን ፡ እምሰማይ ፡ ወርእዩ ፡ ብዙኃ ፡ ደመ ፡ ዘይትከዐው ፡ በዲበ ፡ ምድር ፡ ወኵሎ ፡ ዓመፃ ፡ ዘይትገበር ፡ በዲበ ፡ ምድር ። ወይቤሉ ፡ በበይናቲሆሙ ፡ ቃለ ፡ ጽራኃቲሆሙ ፡ ዕራቃ ፡ ጸርኀት ፡ ምድር ፡ እስከ ፡ አንቀጸ ፡ ሰማይ ። ወይእዜኒ ፡ ለክሙ ፡ ኦቅዱሳነ ፡ ሰማይ ፡ ይሰክዩ ፡ ነፍሳተ ፡ ሰብእ ፡ እንዘ ፡ ይብሉ ፡ አብኡ ፡ ለነ ፡ ፍትሐ ፡ ኀበ ፡ ልዑል ። ወይቤሉ ፡ ለእግዚኦሙ ፡ ለንጉሥ ፡ እስመ ፡ እግዚኦሙ ፡ ለአጋእዝት ፡ ወአምላኮሙ ፡ ለአማልክት ፡ ወንጉሦሙ ፡ ለነገሥት ፡ ወመንበረ ፡ ስብሐቲከ ፡ ውስተ ፡ ኵሎ ፡ ትውልደ ፡ ዓለም ፡ ወስምከ ፡ ቅዱስ ፡ ወስቡሕ ፡ ውስተ ፡ ኵሉ ፡ ትውልደ ፡ ዓለም ፡ ወቡሩክ ፡ ወስቡሕ ፡ አንተ ። ገበርከ ፡ ኵሎ ፡ ወሥልጣነ ፡ ኵሉ ፡ ምስሌከ ፡ ወኵሉ ፡ ክሡት ፡ ቅድሜከ ፡ ወግሁድ ፤ ወአንተ ፡ ትሬኢ ፡ ኵሎ ፡ ወአልቦ ፡ ዘይትከሀል ፡ ይትኀባእ ፡ እምኔከ ። ርኢኬ ፡ ዘገብረ ፡ አዛዝኤል ፡ ዘከመ ፡ መሀረ ፡ ኵሎ ፡ ዓመፃ ፡ በዲበ ፡ ምድር ፡ ወአግሀደ ፡ ኅቡኣተ ፡ ዓለም ፡ እለ ፡ ይትገበራ ፡ በሰማያት ። ወአመረ ፡ ስብዐታተ ፡ ስምያዛ ፡ ዘአንተ ፡ ወሀብኮ ፡ ሥልጣነ ፡ ይኰንን ፡ እለ ፡ ምስሌሁ ፡ ኅቡረ ። ወሖሩ ፡ ኀበ ፡ አዋልደ ፡ ሰብእ ፡ ኅቡረ ፡ ወሰከቡ ፡ ምስሌሆን ፡ ምስለ ፡ እልኩ ፡ አንስት ፡ ወረኵሱ ፡ ወአግሀዱ ፡ ሎን ፡ እሎንተ ፡ ኃጣውአ ። ወአንስትሰ ፡ ወለዳ ፡ ረዓይተ ፡ ወበዝ ፡ መልአት ፡ ኵላ ፡ ምድር ፡ ደመ ፡ ወዐመፃ ። ወይእዜኒ ፡ ናሁ ፡ ይጸርሑ ፡ ነፍሳት ፡ እለ ፡ ሞቱ ፡ ወይሰክዩ ፡ እስከ ፡ አንቀጸ ፡ ሰማይ ፡ ወዐርገ ፡ ገዓሮሙ ፡ ወኢይክሉ ፡ ወፂአ ፡ እምቅድመ ፡ ገጸ ፡ ዐመፃ ፡ ዘይትገበር ፡ በዲበ ፡ ምድር ። ወአንተ ፡ ተአምር ፡ ኵሎ ፡ ዘእንበለ ፡ ይኩን ፤ ወአንተ ፡ ተአምር ፡ ዘንተ ፡ ወዘዚአሆሙ ፡ ወአልቦ ፡ ዘትነግረነ ፡ ወምንት ፡ መፍትው ፡ ንረስዮሙ ፡ በእንተ ፡ ዝንቱ ።} & \\
\textamh{{\Large 10}\ \   ወአሜሃ ፡ ልዑል ፡ ዐቢይ ፡ ወቅዱስ ፡ ተናገረ ፡ ወፈነዎ ፡ ለአርስየላልዩር ፡ ኀበ ፡ ወልደ ፡ ላሜክ ፡ ወይቤሎ ። በሎ ፡ በስመ ፡ ዚአየ ፡ ኅባእ ፡ ርእሰከ ፡ ወአግህድ ፡ ሎቱ ፡ ፍጻሜ ፡ ዘይመጽእ ፡ እስመ ፡ ትትሐጐል ፡ ምድር ፡ ኵላ ፡ ወማየ ፡ አይኅ ፡ ይመጽእ ፡ ሀሎ ፡ ዲበ ፡ ኵላ ፡ ምድር ፡ ወይትኀጐል ፡ ዘሀሎ ፡ ውስቴታ ። ወይእዜኒ ፡ መሀሮ ፡ ከመ ፡ ይንፍጽ ፡ ወይንበር ፡ ዘርኡ ፡ ለኵሉ ፡ ምድር ። ወይቤሎ ፡ ካዕበ ፡ እግዚእ ፡ ለሩፋኤል ፡ እስሮ ፡ ለአዛዝኤል ፡ በእዴሁ ፡ ወእገሪሁ ፡ ወደዮ ፡ ውስተ ፡ ጽልመት ፡ ወአብቅዋ ፡ ለገዳም ፡ እንተ ፡ ሀለወት ፡ በዱዳኤል ፡ ወደዮ ፡ ህየ ። ወደይ ፡ ላዕሌሁ ፡ አዕባነ ፡ ጠዋያተ ፡ ወበሊኃተ ፡ ወክድኖ ፡ ጽልመተ ፡ ወህየ ፡ ይኅድር ፡ ለዓለም ፡ ወክድኖ ፡ ለገጹ ፡ ከመ ፡ እይርአይ ፡ ብርሃነ ። ወበዕለት ፡ ዐባይ ፡ እንተ ፡ ኵነኔ ፡ ከመ ፡ ይትፈነው ፡ ውስተ ፡ ዋዕይ ። ወአሕይዋ ፡ ለምድር ፡ እንተ ፡ አማሰኑ ፡ መላእክት ፡ ወሕይወታ ፡ ለምድር ፡ አይድዕ ፡ ከመ ፡ አሐይዋ ፡ ለምድር ፡ ወኢይትኃጐሉ ፡ ኵሎሙ ፡ ውሉደ ፡ ሰብእ ፡ በምሥጢረ ፡ ኵሉ ፡ ዘቀተሉ ፡ ትጉሃን ፡ ወመሀሩ ፡ ለውሉዶሙ ። ወማሰነት ፡ ኵላ ፡ ምድር ፡ በትምህርተ ፡ ግብሩ ፡ ለአዛዝኤል ፡ ወላዕሌሁ ፡ ጸሐፍ ፡ ኵሎ ፡ ኀጢአተ ። ወለገብርኤል ፡ ይቤሎ ፡ እግዚአብሔር ፡ ሑር ፡ ዲቤሆሙ ፡ ለመንዝራን ፡ ወለምኑናን ፡ ወዲበ ፡ ውሉደ ፡ ዘማ ፡ ወአኅጕሎሙ ፡ ለውሉደ ፡ ዘማ ፡ ወለውሉደ ፡ ትጉሃን ፡ እምሰብእ ፡ ወአውፅኦሙ ፡ ወፈንዎሙ ፡ በበይናቲሆሙ ፡ እሙንቱ ፡ ወለሊሆሙ ፡ በቀትል ፡ ይትኀጐሉ ፡ እስመ ፡ ኑኃ ፡ መዋዕል ፡ አልቦሙ ። ወኵሎሙ ፡ ይስእሉከ ፡ ወኢይከውን ፡ ለአበዊሆሙ ፡ በእንቲአሆሙ ፡ እስመ ፡ ይሴፈዉ ፡ ሕይወተ ፡ ዘለዓለም ፡ ወከመ ፡ ይሕየዉ ፡ እምኔሆሙ ፡ ፭፻ክራማተ ። ወለሚካኤል ፡ ይቤሎ ፡ እግዚአብሔር ፡ አይድዕ ፡ ለስምያዛ ፡ ወለካልኣን ፡ እለ ፡ ምስሌሁ ፡ እለ ፡ ኀብሩ ፡ ምስለ ፡ አንስት ፡ ከመ ፡ ይማስኑ ፡ ምስሌሆን ፡ በኵሉ ፡ ርኵሰ ፡ ዚአሆን ። ሶበ ፡ ይትራገዙ ፡ ኵሉ ፡ ውሉዶሙ ፡ ወሶበ ፡ ይሬእዩ ፡ ሐጕሎሙ ፡ ለፍቁራኒሆሙ ፡ እስሮሙ ፡ ለ፸ትውልድ ፡ በመትሕተ ፡ አውግረ ፡ ምድር ፡ እስከ ፡ ዕለተ ፡ ኵነኔሆሙ ፡ ወተፍጻሜቶሙ ፡ እስከ ፡ ይትፌጸም ፡ ኵነኔ ፡ ዘለዓለመ ፡ ዓለም ። ወበውእቱ ፡ መዋዕል ፡ ይወስድዎሙ ፡ ውስተ ፡ መትሕተ ፡ እሳት ፡ በፃዕር ፡ ወበቤተ ፡ ሞቅሕ ፡ ይትዓፀዉ ፡ ለዓለመ ፡ ዓለም ። ወሶቤሃ ፡ ይውዒ ፡ ወይማስን ፡ እምይእዜ ፡ ምስሌሆሙ ፡ ኅቡረ ፡ ይትአሰሩ ፡ እስከ ፡ ተፍጻሜተ ፡ ትውልደ ፡ ትውልድ ። ወአኅጕሎሙ ፡ ለኵሎሙ ፡ ነፍሳተ ፡ ተውኔት ፡ ወለውሉዶሙ ፡ ለትጉሃን ፡ እስመ ፡ ገፍዕዎሙ ፡ ለሰብእ ። አኅጕል ፡ ኵሎ ፡ ግፍዐ ፡ እምገጸ ፡ ምድር ፡ ወኵሉ ፡ ምግባር ፡ እኩይ ፡ ይኅልቅ ፡ ወያስተርኢ ፡ ተክለ ፡ ጽድቅ ፡ ወርትዕ ፡ ወይከውን ፡ ለበረከት ፡ ግብር ፤ ጽድቅ ፡ ወርትዕ ፡ ለዓለም ፡ በፍሥሓ ፡ ይተክሉ ። ወይእዜኒ ፡ ኵሎሙ ፡ ጻድቃን ፡ ይገንዩ ፡ ወይከውኑ ፡ ሕያዋነ ፡ እስከ ፡ ይወልዱ ፡ ፲፻ ፡ ወኵሎ ፡ መዋዕለ ፡ ውርዙቶሙ ፡ ወሰንበተ ፡ ዚአሆሙ ፡ ይፌጽሙ ፡ በሰላም ። ወበእማንቱ ፡ መዋዕል ፡ ትትገበር ፡ ኵላ ፡ ምድር ፡ በጽድቅ ፡ ወኵለንታሃ ፡ ትተከል ፡ ዕፀወ ፡ ወትመልዕ ፡ በረከተ ። ወኵሎ ፡ ዕፀወ ፡ ኃሤት ፡ ይተክሉ ፡ ዲቤሃ ፡ ወይተክሉ ፡ ዲቤሃ ፡ እውያነ ፡ ወወይን ፡ ዘይተከል ፡ ዲቤሃ ፡ ይገብር ፡ ፍሬ ፡ ለጽጋብ ፤ ወኵሉ ፡ ዘርእ ፡ ዘይዘራእ ፡ ዲቤሃ ፡ አሐቲ ፡ መስፈርት ፡ ትገብር ፡ እልፈ ፡ ወአሐቲ ፡ መስፈርተ ፡ ኤልያስ ፡ ትገብር ፡ ፲ምክያዳተ ፡ ዘይት ። ወአንተ ፡ አንጽሓ ፡ ለምድር ፡ እምኵሉ ፡ ግፍዕ ፤ ወእምኵሉ ፡ ዐመፃ ፤ ወእምኵሉ ፡ ኀጢአት ፤ ወእምኵሉ ፡ ረሲዕ ፤ ወእምኵሉ ፡ ርኵስ ፡ ዘይትገበር ፡ በዲበ ፡ ምድር ፤ አኅልቆሙ ፡ እምዲበ ፡ ምድር ። ወይኩኑ ፡ ኵሉ ፡ ውሉደ ፡ ሰብእ ፡ ጻድቃነ ፡ ወይኩኑ ፡ ኵሉ ፡ አሕዛብ ፡ ያምልኩ ፡ ወይባርኩ ፡ ኪያየ ፡ ወኵሎሙ ፡ ሊተ ፡ ይሰግዱ ። ወትነጽሕ ፡ ምድር ፡ እምኵሉ ፡ ሙስና ፡ ወእምኵሉ ፡ ኃጢአት ፤ ወእምኵሉ ፡ መቅሠፍት ፤ ወእምኵሉ ፡ ፃዕር ፤ ወኢይደግም ፡ ከመ ፡ እፈኑ ፡ ዲቤሃ ፡ አይኃ ፡ ለትውልደ ፡ ትውልድ ፡ ወእስከ ፡ ለዓለም ።} & \\
\textamh{{\Large 11}\ \   ወበእማንቱ ፡ መዋዕል ፡ እፈትሕ ፡ መዛግብተ ፡ በረከት ፡ እለ ፡ በሰማይ ፡ ከመ ፡ አውርዶሙ ፡ ዲበ ፡ ምድር ፡ ዲበ ፡ ግብሮሙ ፡ ወዲበ ፡ ፃማሆሙ ፡ ለውሉደ ፡ ሰብእ ። ሰላም ፡ ወርትዐ ፡ ሱቱፋነ ፡ ይከውኑ ፡ በኵሉ ፡ መዋዕለ ፡ ዓለም ፡ ወበኵሉ ፡ ትውልደ ፡ ዓለም ።} & \\
\textamh{{\Large 12}\ \  ወእምቅድመ ፡ ኵሉ ፡ ነገር ፡ ተከብተ ፡ ሄኖክ ፡ ወአልቦ ፡ ዘየአምሮ ፡ እምውሉደ ፡ ሰብእ ፡ በኀበ ፡ ተከብተ ፡ ወኅበ ፡ ሀሎ ፡ ወምንተ ፡ ኮነ ። ወኵሉ ፡ ግብሩ ፡ ምስለ ፡ ቅዱሳን ፡ ወምስለ ፡ ትጉሃን ፡ በመዋዕለ ፡ ዚአሁ ። ወአነ ፡ ሄኖክ ፡ ኮንኩ ፡ እባርኮ ፡ ለእግዚእ ፡ ዐቢይ ፡ ወለንጉሠ ፡ ዓለም ፡ ወናሁ ፡ ትጉሃን ፡ ይጼውዑኒ ፡ ሊተ ፡ ለሄኖክ ፡ ጸሓፊ ፡ ወይቤሉኒ ። ሄኖክ ፡ ጸሓፌ ፡ ጽድቅ ፡ ሑር ፡ አይድዕ ፡ ለትጉሃነ ፡ ሰማይ ፡ እለ ፡ ኀደጉ ፡ ሰማየ ፡ ልዑለ ፡ ወምቅዋመ ፡ ቅዱሰ ፡ ዘለዓለም ፡ ወምስለ ፡ አንስት ፡ ማሰኑ ፡ ወገብሩ ፡ ዘከመ ፡ ይገብሩ ፡ ውሉደ ፡ ሰብእ ፡ ወነሥኡ ፡ ሎሙ ፡ አንስተ ፡ ወማሰኑ ፡ ዐቢየ ፡ ሙስና ፡ በዲበ ፡ ምድር ። ወኢይከውን ፡ ሎሙ ፡ በዲበ ፡ ምድር ፡ ሰላም ፡ ወኅድገተ ፡ ኀጢአት ፡ እስመ ፡ ኢይትፌሥሑ ፡ በውሉዶሙ ። ቀትለ ፡ ፍቁራኒሆሙ ፡ ይሬእዩ ፡ ወዲበ ፡ ሐጕለ ፡ ውሉዶሙ ፡ ይግዕሩ ፡ ወይስእሉ ፡ ለዓለም ፡ ወኢይከውን ፡ ሎሙ ፡ ምሕረት ፡ ወኢሰላም ።} & \\
\textamh{{\Large 13}\ \  ወሄኖክ ፡ ኀሊፎ ፡ ይቤሎ ፡ ለአዛዝኤል ፡ ኢይከውነከ ፡ ሰላም ፡ ዐቢይ ፡ ኵነኔ ፡ ወፅአ ፡ ላዕሌከ ፡ ይእሥርከ ። ወሣኅት ፡ ወስእለት ፡ ወምሕረት ፡ ኢይከውነከ ፡ በእንተ ፡ ዘመሀርከ ፡ ግፍዐ ፡ ወበእንተ ፡ ኵሉ ፡ ምግባረ ፡ ጽርፈት ፤ ወግፍዕ ፤ ወኀጢአት ፤ ዘአርአይከ ፡ ለውሉደ ፡ ሰብእ ። አሜሃ ፡ ሐዊርየ ፡ ነገርክዎሙ ፡ ለኵሎሙ ፡ ኅቡረ ፤ ወእሙንቱ ፡ ፈርሁ ፡ ኵሎሙ ፡ ፍርሃት ፡ ወረዓድ ፡ ነሥኦሙ ። ወተስእሉኒ ፡ ተዝካረ ፡ ስእለት ፡ ከመ ፡ እጽሐፍ ፡ ሎሙ ፡ ከመ ፡ ይኩኖሙ ፡ ኅድገተ ፡ ወከመ ፡ አነ ፡ አዕርግ ፡ ተዝካረ ፡ ስእለቶሙ ፡ ኀበ ፡ እግዚአብሔር ፡ ሰማየ ። እስመ ፡ ኢይክሉ ፡ እሙንቱ ፡ እምይእዜ ፡ ተናግሮ ፡ ወኢያነሥኡ ፡ አዕይንቲሆሙ ፡ ውስተ ፡ ሰማይ ፡ እምኃፍረተ ፡ አበሳሆሙ ፡ ዘተኰነኑ ። ወአሜሃ ፡ ጸሐፍኩ ፡ ተዝካረ ፡ ስእለቶሙ ፡ ወአስተብቍዖቶሙ ፡ በእንተ ፡ መንፈሶሙ ፡ ወለለ ፡ ፩ምግባሮሙ ፡ ወበእንተ ፡ ዘይስእሉ ፡ ከመ ፡ ይኩኖሙ ፡ ስርየተ ፡ ወኑኀተ ። ወሐዊርየ ፡ ነበርኩ ፡ ዲበ ፡ ማያተ ፡ ዳን ፡ በዳን ፡ እንተ ፡ ይእቲ ፡ እምየማነ ፡ ዓረበ ፡ አርሞን ፡ ወእነብብ ፡ ተዝካረ ፡ ስእለቶሙ ፡ እስከ ፡ ደቀስኩ ። ወናሁ ፡ ሕልም ፡ መጽአኒ ፡ ወራእያት ፡ ዲቤየ ፡ ወድቁ ፡ ወርኢኩ ፡ ራእየ ፡ መቅሠፍት ፡ እንግር ፡ ለውሉደ ፡ ሰማይ ፡ ወእዛለፎሙ ። ወነቂሕየ ፡ መጻእኩ ፡ ኀቤሆሙ ፡ ወኵሎሙ ፡ ጉቡዓነ ፡ ይነብሩ ፡ እንዘ ፡ ይላሕዉ ፡ በኡብልስያኤል ፡ ዘሀለወት ፡ ማእከለ ፡ ሊባኖስ ፡ ወሴኔሴር ፡ እንዘ ፡ ግልቡባን ፡ ገጾሙ ። ወተናገርኩ ፡ በቅድሜሆሙ ፡ ኵሎ ፡ ራእያተ ፡ ዘርኢኩ ፡ በንዋምየ ፡ ወወጠንኩ ፡ እትናገር ፡ ውእተ ፡ ቃላተ ፡ ጽድቅ ፡ ወእዝልፍ ፡ ለትጉሃነ ፡ ሰማይ ።} & \\
\textamh{{\Large 14}\ \   ዝመጽሐፍ ፡ ቃለ ፡ ጽድቅ ፡ ወዘለፋ ፡ ትጉሃን ፡ እለ ፡ እምዓለም ፡ በከመ ፡ አዘዘ ፡ ቅዱስ ፡ ወዐቢይ ፡ በይእቲ ፡ ራእይ ፡ አነ ፡ ርኢኩ ፡ በንዋምየ ፡ ዘአነ ፡ ይእዜ ፡ እነግር ፡ በልሳን ፡ ዘሥጋ ፡ ወበመንፈስየ ፡ ዘወሀበ ፡ ዐቢይ ፡ አፈ ፡ ለሰብእ ፡ ይትናገሩ ፡ ቦቱ ፡ ወይለብዉ ፡ በልብ ። ከመ ፡ ፈጠረ ፡ ወወሀበ ፡ ለሰብእ ፡ ይለብዉ ፡ ቃለ ፡ አእምሮ ፡ ወሊተኒ ፡ ፈጠረ ፡ ወወሀበኒ ፡ እዛለፎሙ ፡ ለትጉሃን ፡ ውሉደ ፡ ሰማይ ። አነ ፡ ስእለተክሙ ፡ ጸሐፍኩ ፡ ወበራእይየ ፡ ከመዝ ፡ ያስተርኢ ፡ እስመ ፡ ስእለትክሙ ፡ ኢትከውነክሙ ፡ ውስተ ፡ ኵሉ ፡ መዋዕለ ፡ ዓለም ፡ ወኵነኔ ፡ ፍጽምት ፡ ላዕሌክሙ ፡ ወኢይከውነክሙ ። ወእምይእዜ ፡ ኢተዓርጉ ፡ ውስተ ፡ ሰማይ ፡ እስከ ፡ ኵሉ ፡ ዓለም ፡ ወውስተ ፡ ምድር ፡ ተነግረ ፡ ይእሥርክሙ ፡ በኵሉ ፡ መዋዕለ ፡ ዓለም ። ወእምቅድመ ፡ ዝንቱ ፡ ርኢክሙ ፡ ኅጕለ ፡ ውሉድክሙ ፡ ፍቁራን ፤ ወአልብክሙ ፡ ጥራያኒሆሙ ፡ አላ ፡ ይወድቁ ፡ ቅድሜክሙ ፡ በሰይፍ ። ወስእለትክሙ ፡ በእንቲአሆሙ ፡ ኢይከውን ፡ ወበእንቲአክሙኒ ፡ ወአንትሙሂ ፡ እንዘ ፡ ትበክዩ ፡ ወታስተበቍዑ ፡ ወኢትትናገሩ ፡ ወኢምንተኒ ፡ ቃለ ፡ እምውስተ ፡ መጽሐፍ ፡ ዘጸሐፍኩ ። ወሊተ ፡ ከመዝ ፡ ራእይ ፡ አስተርአየኒ ፤ ናሁ ፡ ደመናት ፡ በራእይ ፡ ይጼውዑኒ ፤ ወጊሜ ፡ ይጼውዐኒ ፤ ወሩፀተ ፡ ከዋክብት ፡ ወመባርቅት ፡ ያጐጕዑኒ ፡ ወያጽዕቁኒ ፡ ወነፋሳት ፡ በራእይ ፡ ያሰርሩኒ ፡ ወያጔጕዑኒ ። ወነሥኡኒ ፡ ላዐለ ፡ ውስተ ፡ ሰማይ ፡ ወቦእኩ ፡ እስከ ፡ እቀርብ ፡ ኀበ ፡ ጥቅም ፡ ዘኅንጽት ፡ በእብነ ፡ በረድ ፡ ወልሳነ ፡ እሳት ፡ የዓውዳ ፡ ወወጠነ ፡ ያፍርሀኒ ። ወቦእኩ ፡ ውስተ ፡ ልሳነ ፡ እሳት ፡ ወቀረብኩ ፡ ኀበ ፡ ቤት ፡ ዓቢይ ፡ ዘሕኑጽ ፡ በአዕባነ ፡ በረድ ፤ ወአረፍተ ፡ ውእቱ ፡ ቤት ፡ ከመ ፡ ጸፍጸፈ ፡ ሰሌዳ ፡ በአዕባን ፡ ዘእምበረድ ፡ ወምድሩ ፡ በረድ ። ጠፈሩ ፡ ከመ ፡ ሩፀተ ፡ ከዋክብት ፡ ወመባርቅት ፤ ወማእከሎሙ ፡ ኪሩቤል ፡ ዘእሳት ፡ ወሰማዮሙ ፡ ማይ ። ወእሳት ፡ ዘይነድድ ፡ በዓውደ ፡ አረፍቱ ፡ ወኆኅቱ ፡ ይውዒ ፡ በእሳት ። ወቦእኩ ፡ ውስተ ፡ ውእቱ ፡ ቤት ፡ ወምውቅ ፡ ከመ ፡ እሳት ፡ ወቈሪር ፡ ከመ ፡ በረድ ፡ ወኢምንተኒ ፡ ፍግዓ ፡ ወሕይወት ፡ አልቦ ፡ ውስቴቱ ፤ ፍርሃት ፡ ከደነኒ ፡ ወረዓድ ፡ አኀዘኒ ። ወእንዘ ፡ እትሀወክ ፡ ወእርዕድ ፡ ወደቁ ፡ በገጽየ ፡ ወእሬኢ ፡ በራእይ ። ወናሁ ፡ ካልእ ፡ ቤት ፡ ዘየዐቢ ፡ እምዝኩ ፡ ወኵሉ ፡ ኆኅቱ ፡ ርኁት ፡ በቅድሜየ ፡ ወኅኑጽ ፡ በልሳነ ፡ እሳት ። ወበኵሉ ፡ ይራደፍድ ፡ በስብሐት ፡ ወበክብር ፡ ወዕበይ ፡ እስከ ፡ ኢይክል ፡ ዜንዎተክሙ ፡ በእንተ ፡ ስብሐቲሁ ፡ ወበእንተ ፡ ዐበዩ ። ወምድሩሰ ፡ እሳት ፡ ወመልዕልቴሁ ፡ መብረቅ ፡ ወምርዋፀ ፡ ከዋክብት ፡ ወጠፈሩኒ ፡ እሳት ፡ ዘይነድድ ። ወነጸርኩ ፡ ወርኢኩ ፡ ውስቴቱ ፡ መንበረ ፡ ልዑለ ፤ ወራእዩ ፡ ከመ ፡ አስሐትያ ፡ ወክበቡ ፡ ከመ ፡ ፀሐይ ፡ ዘያበርህ ፡ ወቃለ ፡ ኪሩቤል ። ወእመትሕተ ፡ መንበሩ ፡ ዐቢይ ፡ ይወፅእ ፡ አፍላገ ፡ እሳት ፡ ዘይነድድ ፤ ወኢይክሉ ፡ ርእዮቶ ። ወዐቢየ ፡ ስብሐት ፡ ይነብር ፡ ላዕሌሁ ፡ ወዐጽፉሰ ፡ ዘይበርህ ፡ እምፀሐይ ፡ ወይፀዐዱ ፡ እምኵሉ ፡ በረድ ። ወኢይክል ፡ ወኢመኑሂ ፡ እመላእክት ፡ በዊአ ፤ ወራእየ ፡ ገጹ ፡ ለክቡር ፡ ወስቡሕ ፡ ኢይክል ፡ ወኢመኑሂ ፡ ዘሥጋ ፡ ይርአይ ፡ ኪያሁ ። እሳተ ፡ እሳት ፡ ዘይነድድ ፡ በዓውዱ ፡ ወእሳት ፡ ዐቢይ ፡ ይቀውም ፡ ቅድሜሁ ፡ ወአልቦ ፡ ዘይቀርብ ፡ ኀቤሁ ፡ እምእለ ፡ ዓውዱ ፡ ትእልፊተ ፡ ትእልፊት ፡ ቅድሜሁ ፡ ወውእቱሰ ፡ ኢይፈቅድ ፡ ምክረ ፡ ቅድስተ ። ወቅዱሳን ፡ እለ ፡ ይቀርቡ ፡ ኀቤሁ ፡ ኢይርኅቁ ፡ ሌሊተ ፡ ወመዓልተ ፡ ወኢይትአተቱ ፡ እምኔሁ ። ወአነ ፡ ሀለውኩ ፡ እስከ ፡ ዝንቱ ፡ ዲበ ፡ ገጽየ ፡ ግልባቤ ፡ እንዘ ፡ እርዕዱ ፡ ወእግዚእ ፡ በአፉሁ ፡ ጸውዐኒ ፡ ወይቤለኒ ፡ ቅረብ ፡ ዝየ ፡ ሄኖክ ፡ ወለቃልየ ፡ ቅዱስ ። ወአንሥአኒ ፡ ወአቅረበኒ ፡ እስከ ፡ ኆኅት ፤ ወአንሰ ፡ ገጽየ ፡ ታሕተ ፡ እኔጽር ።} & \\
\textamh{{\Large 15}\ \  ወአውሥአኒ ፡ ወይቤለኒ ፡ በቃሉ ፡ ስማዕ ፡ ኢትፍራህ ፡ ሄኖክ ፡ ብእሲ ፡ ጻድቅ ፡ ወጸሓፌ ፡ ጽድቅ ፡ ቅረብ ፡ ዝየ ፡ ወስማዕ ፡ ቃልየ ። ወሑር ፡ በሎሙ ፡ ለትጉሃነ ፡ ሰማይ ፡ እለ ፡ ፈነዉከ ፡ ትስአል ፡ በእንቲአሆሙ ፡ አንትሙ ፡ መፍትው ፡ ትስአሉ ፡ በእንተ ፡ ሰብእ ፡ ወአኮ ፡ ሰብእ ፡ በእንቲአክሙ ። በእንተ ፡ ምንት ፡ ኅደግሙ ፡ ሰማየ ፡ ልዑለ ፡ ወቅዱሰ ፡ ዘለዓለም ፡ ወምስለ ፡ አንስት ፡ ሰከብክሙ ፡ ወምስለ ፡ አዋልደ ፡ ሰብእ ፡ ረኰስክሙ ፡ ወነሣእክሙ ፡ ለክሙ ፡ አንስተ ፡ ወከመ ፡ ውሉደ ፡ ምድር ፡ ገበርክሙ ፡ ወወለድክሙ ፡ ውሉደ ፡ ረዓይተ ። ወአንትሙሰ ፡ መንፈሳውያን ፡ ቅዱሳን ፡ ሕያዋነ ፡ ሕይወት ፡ ዘለዓለም ፡ በዲበ ፡ አንስት ፡ ረኰስክሙ ፡ ወበደመ ፡ ሥጋ ፡ አውለድክሙ ፡ ወበደመ ፡ ሰብእ ፡ ፈተውክሙ ፡ ወገበርክሙ ፡ ከመ ፡ እሙንቱ ፡ ይገብሩ ፡ ሥጋ ፡ ወደመ ፡ እለ ፡ እሙንቱ ፡ ይመውቱ ፡ ወይትኃጐሉ ። ወበእንተዝ ፡ ወሀብክዎሙ ፡ አንስትያ ፡ ከመ ፡ ይዝርኡ ፡ ላዕሌሆን ፡ ወይትወለዱ ፡ ውሉዱ ፡ በላዕሌሆን ፡ ከመ ፡ ከማሁ ፡ ይትገበር ፡ ግብር ፡ በዲበ ፡ ምድር ። ወአንትሙሰ ፡ ቀዳሚ ፡ ኮንክሙ ፡ መንፈሳውያነ ፡ ሕያዋነ ፡ ሕይወት ፡ ዘለዓለም ፡ ዘኢይመውት ፡ ለኵሉ ፡ ትውልደ ፡ ዓለም ። ወበእንተ ፡ ዝንቱ ፡ ኢረሰይኩ ፡ ለክሙ ፡ አንስቲያ ፡ እስመ ፡ መንፈሳውያንሰ ፡ ውስተ ፡ ሰማይ ፡ ማኅደሪሆሙ ። ወይእዜኒ ፡ ረዓይት ፡ እለ ፡ ተወልዱ ፡ እምነፍስት ፡ ወሥጋ ፡ መናፍስተ ፡ እኩያነ ፡ ይሰመዩ ፡ በዲበ ፡ ምድር ፡ ወውስተ ፡ ምድር ፡ ይከውን ፡ ማኅደሪሆሙ ። ወነፍሳት ፡ እኩያን ፡ ወፅኡ ፡ እምሥጋሆሙ ፡ እስመ ፡ እመልዕልት ፡ ተፈጥሩ ፡ እምቅዱሳን ፡ ትጉሃን ፡ ኮኑ ፡ ቀዳሚቶሙ ፡ ወቀዳሚ ፡ መሰረት ፡ መንፈሰ ፡ እኩየ ፡ ይከውኑ ፡ በዲበ ፡ ምድር ፡ ወመንፈሰ ፡ እኩያን ፡ ይሰመዩ ። ወመናፍስተ ፡ ሰማይ ፡ ውስተ ፡ ሰማይ ፡ ይከውን ፡ ማኅደሪሆሙ ፤ ወመናፍስተ ፡ ምድር ፡ እለ ፡ ተወልዱ ፡ በዲበ ፡ ምድር ፡ ውስተ ፡ ምድር ፡ ማኅደሪሆሙ ። ወመንፈሰ ፡ ረዓይት ፡ ደመናተ ፡ እለ ፡ ይገፍዑ ፤ ይማስኑ ፤ ወይወድቁ ፤ ወይትበዐሱ ፤ ወያደቅቁ ፡ ዲበ ፡ ምድር ፤ ወኀዘነ ፡ ይገብሩ ፤ ወኢምንተኒ ፡ ዘይበልዑ ፡ እክለ ፤ ወኢይጸምዑ ፡ ወኢይትዓወቁ ። ወኢይትነሥኡ ፡ እሎንቱ ፡ ነፍሳት ፡ ዲበ ፡ ውሉደ ፡ ሰብእ ፡ ወዲበ ፡ አንስት ፡ እስመ ፡ ወፅኡ ፡ አመ ፡ መዋዕለ ፡ ቀትል ፡ ወሙስና ።} & \\
\textamh{{\Large 16}\ \  ወሞተ ፡ ረዓይትኒ ፡ እንተ ፡ ኀበ ፡ ወፅኡ ፡ መንፈሳት ፡ እምነፍስት ፡ ሥጋሆሙ ፡ ለይኩን ፡ ዘይማስን ፡ ዘእንበለ ፡ ኵነኔ ፤ ከማሁ ፡ ይማስኑ ፡ እስከ ፡ ዕለተ ፡ ኵነኔ ፡ ዐባይ ፡ እምዓለም ፡ ዐቢይ ፡ ይትፌጸም ፡ እምትጉሃን ፡ ወረሲዓን ። ወይእዜኒ ፡ ለትጉሃን ፡ እለ ፡ ፈነዉከ ፡ ትስአል ፡ በእንቲአሆሙ ፡ እለ ፡ ቀዲሙ ፡ በሰማይ ፡ ሀለዉ ። ወእእዜኒ ፡ አንትሙሰ ፡ በሰማይ ፡ ሀለውክሙ ፡ ወኅቡኣት ፡ ዓዲ ፡ እተከሥቱ ፡ ለክሙ ፡ ወምኑነ ፡ ምሥጢረ ፡ አእመርክሙ ፡ ወዘንተ ፡ ዜነውክሙ ፡ ለአንስት ፡ በጽንዐ ፡ ልብክሙ ፡ ወበዝንቱ ፡ ምሥጢር ፡ ያበዝኃ ፡ አንስት ፡ ወሰብእ ፡ እኪተ ፡ በዲበ ፡ ምድር ። በሎሙ ፡ እንከሰ ፡ አልብክሙ ፡ ሰላም ።} & \\
\textamh{{\Large 17}\ \ ወነሥኡኒ ፡ ውስተ ፡ ፩መካን ፡ ኀበ ፡ ሀለዉ ፡ ህየ ፡ ከመ ፡ እሳት ፡ ዘይነድድ ፡ ወሶበ ፡ ይፈቅዱ ፡ ያስተርእዩ ፡ ከመ ፡ ሰብእ ። ወወሰደኒ ፡ ውስተ ፡ መካን ፡ ዘዐውሎ ፡ ወውስተ ፡ ደብር ፡ ዘከተማ ፡ ርእሱ ፡ ይበጽሕ ፡ ውስተ ፡ ሰማይ ። ወርኢኩ ፡ መካናተ ፡ ብሩሃነ ፡ ወነጐድጓድ ፡ ውስተ ፡ አጽናፍ ፤ ኀበ ፡ ዕመቁ ፡ ቀስተ ፡ እሳት ፤ ወሐፅ ፤ ወምጕንጳቲሆሙ ፤ ወሰይፈ ፡ እሳት ፤ ወመባርቅት ፡ ኵሉ ። ወነሥኡኒ ፡ እስከ ፡ ማየ ፡ ሕይወት ፡ ዘይትነገር ፡ ወእስከ ፡ እሳተ ፡ ዓረብ ፡ ዘውእቱ ፡ ይእኅዝ ፡ ኵሎ ፡ ዕርበተ ፡ ፀሐይ ። ወመጻእኩ ፡ እስከ ፡ ፈለገ ፡ እሳት ፡ ዘይውኅዝ ፡ እሳቱ ፡ ከመ ፡ ማይ ፡ ወይትከዐው ፡ ውስተ ፡ ባሕር ፡ ዐቢይ ፡ ዘመንገለ ፡ ዓረብ ። ወርኢኩ ፡ ኵሎ ፡ ዐበይተ ፡ አፍላገ ፡ ወእስከ ፡ ዐቢይ ፡ ጽልመት ፡ በጻሕኩ ፡ ወሖርኩ ፡ ኀበ ፡ ኵሉ ፡ ዘሥጋ ፡ ያንሶሱ ። ወርኢኩ ፡ አድባረ ፡ ቆባራት ፡ እለ ፡ ክረምት ፡ ወምክዓወ ፡ ማይ ፡ ዘኵሉ ፡ ቀላይ ። ወርኢኩ ፡ አፉሆሙ ፡ ለኵሎሙ ፡ አፍላገ ፡ ምድር ፡ ወአፉሃ ፡ ለቀላይ ።} & \\
\textamh{{\Large 18}\ \ ወርኢኩ ፡ መዛግብተ ፡ ኵሉ ፡ ነፋሳት ፡ ወርኢኩ ፡ ከመ ፡ ቦሙ ፡ አሰርገወ ፡ ኵሎ ፡ ፍጥረተ ፡ ወመሰረታቲሃ ፡ ለምድር ። ወርኢኩ ፡ እብነ ፡ ማእዘንተ ፡ ምድር ፡ ወርኢኩ ፡ ፬ነፋሳተ ፡ እለ ፡ ይፀውርዋ ፡ ለምድር ፡ ወለጽንዐ ፡ ሰማይ ። ወርኢኩ ፡ ከመ ፡ ነፋሳት ፡ ይረብብዋ ፡ ለልዕልና ፡ ሰማይ ፤ ወእሙንቱ ፡ ይቀውሙ ፡ ማእከለ ፡ ሰማይ ፡ ወምድር ፡ እሙንቱ ፡ ውእቶሙ ፡ አዕማደ ፡ ሰማይ ። ወርኢኩ ፡ ነፋሳተ ፡ እለ ፡ ይመይጥዋ ፡ ለሰማይ ፡ እለ ፡ ያዐርቡ ፡ ለክበበ ፡ ፀሐይ ፡ ወኵሎ ፡ ከዋክብተ ። ወርኢኩ ፡ ዘዲበ ፡ ምድር ፡ ነፋሳተ ፡ ዘይፀውሩ ፡ ደመናተ ፤ ወርኢኩ ፡ ፍናወ ፡ መላእክት ፤ ርኢኩ ፡ ውስተ ፡ ጽንፈ ፡ ምድር ፡ ጽንዐ ፡ ዘሰማይ ፡ መልዕልተ ። ወኅለፍኩ ፡ መንገለ ፡ አዜብ ፡ ወይነድድ ፡ መዓልተ ፡ ወሌሊተ ፡ ኀበ ፡ ፯አድባር ፡ ዘእምእብን ፡ ክቡር ፡ ፫መንገለ ፡ ጽባሕ ፡ ወ፫መንገለ ፡ አዜብ ። ወመንገለ ፡ ጽባሕሰ ፡ ዘእምእብነ ፡ ኅብር ፡ ወ፩ሰ ፡ ዘእምእብነ ፡ ባሕርይ ፡ ወ፩ኒ ፡ ዘእምእብነ ፡ ፈውስ ፡ ወዘመንገለ ፡ አዜብ ፡ እምነ ፡ እብን ፡ ቀይሕ ። ወማእከላይሰ ፡ ይጐድእ ፡ እስከ ፡ ሰማይ ፡ ከመ ፡ መንበሩ ፡ ለእግዚአብሔር ፡ ዘእምእብነ ፡ ፔካ ፡ ወድማሁ ፡ ለመንበር ፡ ዘእምእብነ ፡ ሰንፔር ። ወእሳተ ፡ ዘይነድድ ፡ ርኢኩ ፡ ወዘሀሎ ፡ ውስተ ፡ ኵሉ ፡ አድባር ። ወርኢኩ ፡ ህየ ፡ መካነ ፡ ማዕዶቱ ፡ ለዐቢይ ፡ ምድር ፡ ህየ ፡ ይትጋብኡ ፡ ማያት ። ወርኢኩ ፡ ንቅዐተ ፡ ምድር ፡ ዕሙቀ ፡ በአእማዲሁ ፡ ለእሳተ ፡ ሰማይ ፡ ወርኢኩ ፡ በውስቴቶሙ ፡ አዕማደ ፡ ሰማይ ፡ ዘእ ፡ ሳት ፡ ዘይወርዱ ፡ ወአልቦሙ ፡ ኍልቍ ፡ ወኢመንገለ ፡ መልዕልት ፡ ወኢመንገለ ፡ ዕመቅ ። ወዲበ ፡ ውእቱ ፡ ንቅዐት ፡ ርኢኩ ፡ መካነ ፡ ወኢጽንዐ ፡ ሰማይ ፡ ላዕሌሁ ፡ ወኢመሠረተ ፡ ምድር ፡ በታሕቱ ፡ ወኢማይ ፡ አልቦ ፡ ላዕሌሁ ፡ ወኢአዕዋፍ ፡ አላ ፡ መካነ ፡ በድው ፡ ውእቱ ። ወግሩመ ፡ ርኢኩ ፡ በህየ ፡ ፯ከዋክብተ ፡ ከመ ፡ ዐበይት ፡ አድባር ፡ ዘይነድዱ ፡ ወከመ ፡ መንፈስ ፡ ዘይሴአለኒ ። ይቤ ፡ መልአክ ፡ ዝውእቱ ፡ መካነ ፡ ተፍጻሜቱ ፡ ለሰማይ ፡ ወለምድር ፤ ቤተ ፡ ሞቅሕ ፡ ኮኖሙ ፡ ዝንቱ ፡ ለከዋክብተ ፡ ሰማይ ፡ ወለኀይለ ፡ ሰማይ ። ወከዋክብት ፡ እለ ፡ ያንኰረኵሩ ፡ ዲበ ፡ እሳት ፡ ወእሉ ፡ ውእቶሙ ፡ እለ ፡ ኀለፉ ፡ ትእዛዘ ፡ እግዚአብሔር ፡ እምቅድመ ፡ ጽባሖሙ ፡ እስመ ፡ ኢመጽኡ ፡ በጊዜሆሙ ። ወተምዕዖሙ ፡ ወአሠሮሙ ፡ እስከ ፡ ጊዜ ፡ ተፍጻሜተ ፡ ኀጢአቶሙ ፡ በዓመተ ፡ ምሥጢር ።} & \\
\textamh{{\Large 19}\ \ ወይቤለኒ ፡ ኡርኤል ፡ በዝየ ፡ ተደሚሮሙ ፡ መላ ፡ እክት ፡ ምስለ ፡ አንስት ፡ ይቀውሙ ፡ መናፍስቲሆሙ ፡ ወብዙኃ ፡ ራእየ ፡ ከዊኖሙ ፡ አርኰስዎሙ ፡ ለሰብእ ፡ ወያስሕትዎሙ ፡ ለሰብእ ፡ ከመ ፡ ይሡዑ ፡ ለአጋንንት ፡ ከመ ፡ አማልክት ፡ እስመ ፡ በዕለት ፡ ዐባይ ፡ ኵነኔ ፡ በዘይትኴነኑ ፡ እስከ ፡ ይትፌጸሙ ። ወአንስቲያሆሙኒ ፡ አስሒቶን ፡ መላእክተ ፡ ሰማይ ፡ ከመ ፡ ሰላማውያን ፡ ይከውና ። ወአነ ፡ ሄኖክ ፡ ርኢኩ ፡ አርአያ ፡ ባሕቲትየ ፡ አጽናፈ ፡ ኵሉ ፡ ወአልቦ ፡ ዘርእየ ፡ እምሰብእ ፡ ከመ ፡ አነ ፡ ርኢኩ ።} & \\
\textamh{{\Large 20}\ \ ወዝንቱ ፡ ውእቱ ፡ አስማ ፡ ቲሆሙ ፡ ለእለ ፡ ይተግሁ ፡ ቅዱሳን ፡ መላእክት ። ኡርኤል ፡ ፩እምነ ፡ መላእክት ፡ ቅዱሳን ፡ እስመ ፡ ዘረዓም ፡ ወዘረዓድ ። ሩፋኤል ፡ ፩እምነ ፡ መላእክት ፡ ቅዱሳን ፡ ዘመናፍስተ ፡ ሰብእ ። ራጉኤል ፡ ፩እምነ ፡ መላእክት ፡ ቅዱሳን ፡ ዘይትቤቀሎ ፡ ለዓለም ፡ ወለብርሃናት ። ሚካኤል ፡ ፩እምነ ፡ መላእክት ፡ ቅዱሳን ፡ እስመ ፡ በዲበ ፡ ሠናይቱ ፡ ለሰብእ ፡ ተአዛዚ ፡ ዲበ ፡ ሕዝብ ። ሰረቃኤል ፡ ፩እምነ ፡ መላእክት ፡ ቅዱሳን ፡ ዘዲበ ፡ መናፍስተ ፡ እጓለ ፡ እመሕያው ፡ ዘመናፍስተ ፡ ያኅጥኡ ። ገብርኤል ፡ ፩እምነ ፡ መላእክት ፡ ቅዱሳን ፡ ዘዲበ ፡ አኪስት ፡ ወዘዲበ ፡ ገነት ፡ ወዘኪሩቤል ።} & \\
\textamh{{\Large 21}\ \ ወኦድኩ ፡ እስከ ፡ መካን ፡ ኀበ ፡ አልቦ ፡ ዘይትገበር ። ወርኢኩ ፡ በህየ ፡ ግብረ ፡ ግሩመ ፡ ኢሰማየ ፡ ልዑለ ፡ ወኢምድረ ፡ ሱርርተ ፡ አላ ፡ መካነ ፡ በድው ፡ ዘድልው ፡ ወግፉም ። ወህየ ፡ ርኢኩ ፡ ፯ከዋክብተ ፡ ሰማይ ፡ እሱራን ፡ በላዕሌሁ ፡ ኅብረ ፡ ከመ ፡ አድባር ፡ ዐበይት ፡ ወከመ ፡ ዘእሳት ፡ እንዘ ፡ ይነድዱ ። ውእተ ፡ ጊዜ ፡ እቤ ፡ በእንተ ፡ አይ ፡ ኅጢአት ፡ ተአሥሩ ፡ ወበእንተ ፡ ምንት ፡ ዝየ ፡ ተገድፉ ። ወይቤለኒ ፡ ኡርኤል ፡ ፩እምነ ፡ መላእክት ፡ ቅዱሳን ፡ ዘምስሌየ ፡ ውእቱ ፡ ይመርሀኒ ፡ ወይቤ ፡ ሄኖክ ፡ በእንተ ፡ መኑ ፡ ትሴአል ፡ ወበእንተ ፡ መኑ ፡ ትጤይቅ ፡ ወትስእል ፡ ወትጽህቅ ። እሉ ፡ ውእቶሙ ፡ እምነ ፡ ከዋክብት ፡ እለ ፡ ኀለፉ ፡ ትእዛዘ ፡ እግዚአብሔር ፡ ልዑል ፡ ወተአሥሩ ፡ በዝየ ፡ እስከ ፡ ይትፌጸም ፡ ትእልፊተ ፡ ዓለም ፡ ኍልቈ ፡ መዋዕለ ፡ ኃጢአቶሙ ። ወእምህየ ፡ ሖርኩ ፡ ካልአ ፡ መካነ ፡ እምዝ ፡ ዘይጌርም ፡ ወርኢኩ ፡ ግብረ ፡ ግሩመ ፡ እሳት ፡ ዐቢይ ፡ በህየ ፡ ዘይነድድ ፡ ወያንበለብል ፡ ወመምተርት ፡ ቦቱ ፡ ወሰኑ ፡ እስከ ፡ ቀላይ ፡ ፍጹም ፡ አዕማደ ፡ እሳት ፡ ዐበይት ፡ ዘያወርድዎሙ ፡ ወኢአምጣኖ ፡ ወኢዕበዮ ፡ እክህልኩ ፡ ነጽሮ ፡ ወስእንኩ ፡ ነጽሮ ፡ ዐይኑ ። ውእተ ፡ ጊዜ ፡ እቤ ፡ እፎ ፡ ግሩም ፡ ዝንቱ ፡ መካን ፡ ወሕማም ፡ ለነጽሮ ። ውእተ ፡ ጊዜ ፡ አውሥአኒ ፡ ኡርኤል ፡ ፩እመላእክት ፡ ቅዱሳን ፡ ዘምስሌየ ፡ ሀለወ ፡ አውሥአኒ ፡ ወይቤለኒ ፡ ሄኖክ ፡ ምንት ፡ ውእቱ ፡ ፍርሃትከ ፡ ከመዝ ፡ ወድንጋፄከ ፡ በእንተዝ ፡ ግሩም ፡ መካን ፡ ወቅድመ ፡ ገጹ ፡ ለዝ ፡ ሕማም ። ወይቤለኒ ፡ ዝመካን ፡ ቤተ ፡ ሞቅሖሙ ፡ ለመላእክት ፡ ወበህየ ፡ ይትአኅዙ ፡ እስከ ፡ ለዓለም ።} & \\
\textamh{{\Large 22}\ \ ወእምህየ ፡ ሖርኩ ፡ ካልአ ፡ መካነ ፡ ወአርአየኒ ፡ በምዕራብ ፡ ደብረ ፡ ዐቢየ ፡ ወነዋኀ ፡ ወኰኵሕ ፡ ጽኑዕ ፡ ወ፬መካናት ፡ ሠናያት ። ወበውስቴቱ ፡ ዘቦቱ ፡ ዕሙቅ ፡ ወርኂብ ፡ ወልሙጽ ፡ ጥቀ ፡ ከመ ፡ ልሙጽ ፡ ዘያንኰረኵር ፡ ወዕሙቅ ፡ ወጽልመት ፡ ለነጽሮ ። ውእተ ፡ ጊ ፡ ዜ ፡ አውሥአኒ ፡ ሩፋኤል ፡ ፩እመላእክት ፡ ቅዱሳን ፡ ዘሀለወ ፡ ምስሌየ ፡ ወይቤለኒ ፡ እላ ፡ መካናት ፡ ሠናያት ፡ ከመ ፡ ይትጋብኡ ፡ ዲቤሆን ፡ መናፍስት ፡ ነፍሶሙ ፡ ለምውታን ፡ ሎሙ ፡ እሎንቱ ፡ ተፈጥሩ ፡ ዝየ ፡ ያስተጋብኡ ፡ ኵሎ ፡ ነፍሰ ፡ ውሉደ ፡ ሰብእ ። ወእሙንቱ ፡ መካናት ፡ ኀበ ፡ ያነብርዎሙ ፡ ገብሩ ፡ እስከ ፡ ዕለተ ፡ ኵነኔሆሙ ፡ ወእስከ ፡ አመ ፡ ዕድሜሆሙ ፤ ወዕድሜ ፡ ውእቱ ፡ ዐቢይ ፡ እስከ ፡ አመ ፡ ኵነኔ ፡ ዐቢይ ፡ በላዕሌሆሙ ። ወርኢኩ ፡ መናፍስተ ፡ ውሉደ ፡ ሰብእ ፡ እንዘ ፡ ምውታን ፡ ውእቶሙ ፡ ወቃሎሙ ፡ ይበጽሕ ፡ እስከ ፡ ሰማይ ፡ ወይሰኪ ። ይእተ ፡ ጊዜ ፡ ተስእልክዎ ፡ ለሩፋኤል ፡ መልአክ ፡ ዘሀሎ ፡ ምስሌየ ፡ ወእቤሎ ፡ ዝመንፈስ ፡ ዘመኑ ፡ ውእቱ ፡ ዘከመዝ ፡ ቃሉ ፡ ይበጽሕ ፡ ወይሰኪ ። ወአውሥአኒ ፡ ወይቤለኒ ፡ እንዘ ፡ ይብል ፡ ዝንቱ ፡ መንፈስ ፡ ውእቱ ፡ ዘይወፅእ ፡ እምአቤል ፡ ዘቀተሎ ፡ ቃየል ፡ እኁሁ ፡ ወይሰኪ ፡ ኪያሁ ፡ እስከ ፡ ሶበ ፡ ይትኃጐል ፡ ዘርኡ ፡ እምገጸ ፡ ምድር ፡ ወእምዘርአ ፡ ሰብእ ፡ ይማስን ፡ ዘርኡ ። ወበእንተዝ ፡ ውእተ ፡ ጊዜ ፡ ተስእልኩ ፡ በእንቲአሁ ፡ ወበእንተ ፡ ኵነኔ ፡ ኵሉ ፡ ወእቤ ፡ በእንተ ፡ ምንት ፡ ተፈልጠ ፡ ፩እምነ ፡ ፩ ። ወአውሥአኒ ፡ ወይቤለኒ ፡ እሉ ፡ ፫ተገብሩ ፡ ከመ ፡ ይፍልጡ ፡ መንፈሶሙ ፡ ለምውታን ፤ ወከመዝ ፡ ተፈልጠ ፡ ነፍሶሙ ፡ ለጻድቃን ፤ ዝውእቱ ፡ ነቅዐ ፡ ማይ ፡ በላዕሌሁ ፡ ብርሃን ። በከመ ፡ ከማሁ ፡ ተፈጥረ ፡ ለኃጥአን ፡ ሶበ ፡ ይመውቱ ፡ ወይትቀበሩ ፡ ውስተ ፡ ምድር ፡ ወኵነኔ ፡ ኢኮነ ፡ በላዕሌሆሙ ፡ በሕይወቶሙ ። ወበዝየ ፡ ይትፈለጣ ፡ ነፍሳቲሆሙ ፡ ዲበ ፡ ዛቲ ፡ ዐባይ ፡ ፃዕር ፡ እስከ ፡ አመ ፡ ዐባይ ፡ ዕለት ፡ እንተ ፡ ኵነኔ ፡ ወመቅሠፍት ፡ ወፃዕር ፡ ለእለ ፡ ይረግሙ ፡ እስከ ፡ ለዓለም ፡ ወበቀል ፡ ለነፍሶሙ ፡ ወበህየ ፡ የአሥሮሙ ፡ እስከ ፡ ለዓለም ። ወእመኒ ፡ ውእቱ ፡ እምቅድመ ፡ ዓለም ፡ ወከመዝ ፡ ተፈልጠ ፡ ለነፍሶሙ ፡ ለእለ ፡ ይሰክዩ ፡ ወለእለ ፡ ያርእዩ ፡ በእንተ ፡ ሕጕለት ፡ አመ ፡ ተቀትሉ ፡ በመዋዕለ ፡ ኃጥኣን ። ከመዝ ፡ ተፈጥረ ፡ ለነፍሶሙ ፡ ለሰብእ ፡ እለ ፡ ኢኮኑ ፡ ጻድቃነ ፡ አላ ፡ ኃጥኣነ ፡ እለ ፡ ፍጹማነ ፡ አበሳ ፡ ወምስለ ፡ አባስያን ፡ ይከውኑ ፡ ከማሆሙ ፡ ወነፍሶሙሰ ፡ ኢትትቀተል ፡ በዕለተ ፡ ኵነኔ ፡ ወኢይትነሥኡ ፡ እምዝየ ። ውእተ ፡ ጊዜ ፡ ባረክዎ ፡ ለእግዚአ ፡ ስብሐት ፡ ወእቤ ፡ ቡሩክ ፡ ውእቱ ፡ እግዚእየ ፡ እግዚአ ፡ ስብሐት ፡ ወጽድቅ ፡ ዘኵሎ ፡ ይመልክ ፡ እስከ ፡ ለዓለም ።} & \\
\textamh{{\Large 23}    ወእምህየ ፡ ሖርኩ ፡ ካልአ ፡ መካነ ፡ መንገለ ፡ ዓረብ ፡ እስከ ፡ አጽናፈ ፡ ምድር ። ወርኢኩ ፡ እሳተ ፡ ዘይነድድ ፡ ወይረውፅ ፡ እንዘ ፡ ኢያዐርፍ ፡ ወኢይነትግ ፡ እምሩጸቱ ፡ መዓልተ ፡ ወሌሊተ ፡ አላ ፡ ከማሁመ ። ወተስእልኩ ፡ እንዘ ፡ እብል ፡ ዝንቱ ፡ ምንት ፡ ውእቱ ፡ ዘአልቦ ፡ ዕረፍት ። ውእተ ፡ ጊዜ ፡ አውሥአኒ ፡ ራጉኤል ፡ ፩እምነ ፡ መላእክት ፡ ቅዱሳን ፡ ዘሀሎ ፡ ምስሌየ ፡ ወይቤለኒ ፡ ዝንቱ ፡ ዘርኢከ ፡ ሩጸተ ፡ ዘመንገለ ፡ ዐረብ ፡ እሳት ፡ ዘይነድድ ፡ ውእቱ ፡ ኵሉ ፡ ብርሃናተ ፡ ሰማይ ።} & \\
\textamh{{\Large 24}    ወእምህየ ፡ ሖርኩ ፡ ካልአ ፡ መካነ ፡ ምድር ፡ ወአርአየኒ ፡ ደብረ ፡ እሳት ፡ ዘያንበለብል ፡ መዓልተ ፡ ወሌሊተ ። ወሖርኩ ፡ መንገሌሁ ፡ ወርኢኩ ፡ ፯አድባረ ፡ ክቡራነ ፡ ወኵሉ ፡ ፩እምነ ፡ ፩እንዘ ፡ ይትዌለጥ ፡ ወአዕባነ ፡ ክቡራነ ፡ ወሠናያነ ፡ ወኵሉ ፡ ክቡር ፤ ወስብሕ ፡ ራእዮሙ ፡ ወሠናይ ፡ ገጾሙ ፡ ፫መንገለ ፡ ጽባሕ ፡ ወጽኑዓን ፡ ፩ዲበ ፡ ፩ወ፫መንገለ ፡ ሰሜን ፡ ወጽኑዓን ፡ ፩ዲበ ፡ ፩ወቈላተ ፡ ዕሙቃተ ፡ ወጠዋያተ ፡ አሐቲ ፡ ለአሐቲ ፡ እይትቃረባ ። ወሳብዕ ፡ ደብር ፡ ማእከሎሙ ፡ ለእሎንቱ ፡ ወኑኆሙሰ ፡ ይትማሰሉ ፡ ኵሎሙ ፡ ከመ ፡ መንበረ ፡ አትሮንስ ፡ ወየዐውድዎ ፡ ዕፀወ ፡ መዐዛ ። ወሀሎ ፡ ውስቴቶሙ ፡ ዕፅ ፡ አልቦ ፡ ግሙራ ፡ አመ ፡ ፄነወኒ ፡ ወኢ ፡ ፩እምውስቴቶሙ ፡ ወባዕዳንሂ ፡ ዘከማሁ ፡ ኢኮነ ፡ ዘይምዕዝ ፡ እምኵሉ ፡ መዐዛ ፡ ወቈጽሉ ፡ ወጽጌሁ ፡ ወዕፁ ፡ ኢይጸመሂ ፡ ለዓለም ፡ ወፍሬሁኒ ፡ ሠናይ ፤ ወፍሬሁሰ ፡ ከመ ፡ አስካለ ፡ በቀልት ። ወውእተ ፡ ጊዜ ፡ እቤ ፡ ነዋ ፡ ዝንቱ ፡ ሠናይ ፡ ዕፅ ፡ ወሠናይ ፡ ለርእይ ፡ ወአዳም ፡ ቈጽሉ ፡ ወፍሬሆኒ ፡ ሞገስ ፡ ጥቀ ፡ ለርእየ ፡ ገጽ ። ወውእተ ፡ ጊዜ ፡ አውሥአኒ ፡ ሚካኤል ፡ ፩እምነ ፡ መላእክት ፡ ቅዱሳን ፡ ወክቡራን ፡ ዘምስሌየ ፡ ሀሎ ፡ ውእቱ ፡ ዘዲቤሆሙ ።} & \\
\textamh{{\Large 25}\ \ ወይቤለኒ ፡ ሄኖክ ፡ ምንተ ፡ ትሴአለኒ ፡ በእንተ ፡ መዐዛሁ ፡ ለዝ ፡ ዕፅ ፡ ወትጤይቅ ፡ ከመ ፡ ታእምር ። ወውእተ ፡ ጊዜ ፡ አውሣእክዎ ፡ አነ ፡ ሄኖክ ፡ እንዘ ፡ እብል ፡ በእንተ ፡ ኵሉ ፡ እፈቅድ ፡ አእምር ፡ ወፈድፋደሰ ፡ በእንተዝ ፡ ዕፅ ። ወአውሥአኒ ፡ እንዘ ፡ ይብል ፡ ዝንቱ ፡ ደብር ፡ ዘርኢከ ፡ ነዊኀ ፡ ዘርእሱ ፡ ይመስል ፡ መንበሮ ፡ ለእግዚእ ፡ መንበሩ ፡ ውእቱ ፡ ኀበ ፡ ይነብር ፡ ቅዱስ ፡ ወዐቢይ ፡ እግዚአ ፡ ስብሐት ፡ ንጉሥ ፡ ዘለዓለም ፡ ሶበ ፡ ይወርድ ፡ የሐውጻ ፡ ለምድር ፡ በሠናይ ። ወዝንቱኒ ፡ ዕፀ ፡ መዐዛ ፡ ሠናይ ፡ ወኢ፩ ፡ ዘሥጋ ፡ አልቦ ፡ ሥልጣን ፡ ከመ ፡ ይግሥሦ ፡ እስከ ፡ አመ ፡ ዐቢይ ፡ ኵነኔ ፡ አመ ፡ ይትቤቀል ፡ ኵሎ ፡ ወይትፌጸም ፡ እስከ ፡ ለዓለም ፡ ዝኩ ፡ ለጻድቃን ፡ ወለትሑታን ፡ ይትወሀብ ። እምፍሬ ፡ ዚአሁ ፡ ይትወሀብ ፡ ለኅሩያን ፡ ሕይወት ፡ መንገለ ፡ መስዕ ፡ ይተከል ፡ ውስተ ፡ መካን ፡ ቅዱስ ፡ መንገለ ፡ ቤቱ ፡ ለእግዚእ ፡ ንጉሥ ፡ ዘለዓለም ። ውእተ ፡ ጊዜ ፡ ይትፌሥሑ ፡ በፍሥሓ ፡ ወይትሐሠዩ ፡ ውስተ ፡ ቅዱስ ፡ ያበውኡ ፡ ሎቱ ፡ መዐዛ ፡ በበአዕፅምቲሆሙ ፡ ወሕይወተ ፡ ብዙኃ ፡ የሐይዉ ፡ በዲበ ፡ ምድር ፡ በከመ ፡ ሐይዉ ፡ አበዊከ ፡ ወበመዋዕሊሆሙ ፡ ኀዘን ፡ ወሕማም ፡ ወጻማ ፡ ወመቅሠፍት ፡ እይገሥሦሙ ። ውእተ ፡ ጊዜ ፡ ባረክዎ ፡ ለእግዚአ ፡ ስብሐት ፡ ንጉሥ ፡ ዘለዓለም ፡ እስመ ፡ አስተዳለወ ፡ ከመዝ ፡ ለሰብእ ፡ ጻድቃን ፡ ወከመዝ ፡ ፈጠረ ፡ ወይቤ ፡ የሀብዎሙ ።} & \\
\textamh{{\Large 26}\ \  ወእምህየ ፡ ሖርኩ ፡ ማእከለ ፡ ምድር ፡ ወርኢኩ ፡ መካነ ፡ ቡሩከ ፡ ወጥሉለ ፡ ዘቦቱ ፡ አዕፅቅ ፡ ዘይነብር ፡ ወይሠርፅ ፡ እምዕፅ ፡ ዘተመትረ ። ወበህየ ፡ ርኢኩ ፡ ደብረ ፡ ቅዱሰ ፡ ወመትሕተ ፡ ደብር ፡ ማይ ፡ ዘመንገለ ፡ ጽባሑ ፡ ወውኅዘቱ ፡ መንገለ ፡ ሰሜን ። ወርኢኩ ፡ መንገለ ፡ ጽባሕ ፡ ካልአ ፡ ደብረ ፡ ዘይነውኅ ፡ ከመዝ ፡ ወማእከሎሙ ፡ ቈላ ፡ ዕሙቅ ፡ ወአልቦ ፡ ራኅብ ፡ ወላቲኒ ፡ የሐውር ፡ ማይ ፡ መንገለ ፡ ደብር ፡ ወመንገለ ፡ ዐረቡ ፡ ለዝ ፡ ካልእ ፡ ደብር ፡ ወይቴሐቶ ፡ ሎቱ ፡ ወአልቦ ፡ ኑኅ ፡ ወቈላ ፡ ታሕቱ ፡ ማእከሎሙ ፡ ወካልአት ፡ ቈላት ፡ ዕሙቃት ፡ ወይቡሳት ፡ መንገለ ፡ ጽንፈ ፡ ሠለስቲሆሙ ። ወኵሉ ፡ ቈላቱ ፡ ዕሙቃት ፡ ወአልቦን ፡ ራኅብ ፡ እምኰኵሕ ፡ ጽኑዕ ፤ ወዕፅ ፡ ይተከል ፡ በላዕሌሆሙ ። ወአንከርኩ ፡ በእንተ ፡ ኰኵሕ ፡ ወአንከርኩ ፡ በእንተ ፡ ቈላ ፡ ወጥቀ ፡ አንከርኩ ።} & \\
\textamh{{\Large 27}\ \  ውእተ ፡ ጊዜ ፡ እቤ ፡ በእንተ ፡ ምንት ፡ ዛቲ ፡ ምድር ፡ ቡርክት ፡ ወኵለንታሃ ፡ ዕፀወ ፡ ምልዕት ፡ ወዛቈላ ፡ ርግምት ፡ ማእከሎሙ ። ውእተ ፡ ጊዜ ፡ አውሥአኒ ፡ ኡራኤል ፡ ፩እመላእክት ፡ ቅዱሳን ፡ ዘሀሎ ፡ ምስሌየ ፡ ወይቤለኒ ፡ ዛቈላ ፡ ርግምት ፡ ለርጉማን ፡ እስከ ፡ ለዓለም ፤ ዝየ ፡ ይትጋብኡ ፡ ኵሎሙ ፡ እለ ፡ ይብሉ ፡ በአፉሆሙ ፡ ላዕለ ፡ እግዚአብሔር ፡ ቃለ ፡ ዘኢይደሉ ፡ ወበእንተ ፡ ስብሐተ ፡ ዚአሁ ፡ ይትናገሩ ፡ ዕፁባተ ፡ ዝየ ፡ ያስተጋብእዎሙ ፡ ወዝየ ፡ ምኵናኖሙ ። ወበደኃሪ ፡ መዋዕል ፡ ይከውን ፡ ላዕሌሆሙ ፡ አርአያ ፡ ኵነኔ ፡ ዘበጽድቅ ፡ በቅድመ ፡ ጻድቃን ፡ ለዓለም ፡ ኵሎ ፡ መዋዕለ ፡ በዝየ ፡ ይባርክዎ ፡ መሀርያን ፡ ለእግዚአ ፡ ስብሐት ፡ ንጉሥ ፡ ዘለዓለም ። ወበመዋዕለ ፡ ኵነኔሆሙ ፡ ይባርክዎ ፡ በምሕረት ፡ በከመ ፡ ከፈሎሙ ። ውእተ ፡ ጊዜ ፡ አነኒ ፡ ባረክዎ ፡ ለእግዚአ ፡ ስብሐት ፡ ወነገርኩ ፡ ሎቱ ፡ ወዘከርኩ ፡ በከመ ፡ ይደሉ ፡ ለዕበዩ ።} & \\
\textamh{{\Large 28}\ \  ወእምህየ ፡ ሖርኩ ፡ መንገለ ፡ ጽባሕ ፡ ማእከላ ፡ ለደብረ ፡ መድበራ ፡ ወርኢክዎ ፡ ገዳመ ፡ ባሕቲቶ ። ወባሕቱ ፡ ምሉእ ፡ ዕፀወ ፡ እምነ ፡ ዝንቱ ፡ ዘርእ ፡ ወማይ ፡ እምላዕሉ ፡ ይፈለፍል ፡ በላዕሉ ። ያስተርኢ ፡ አስራብ ፡ ከመ ፡ ብዙኅ ፡ ዘይሰርብ ፡ ከመ ፡ መንገለ ፡ መስዕ ፡ መንገለ ፡ ዐረብ ፡ ወእምኵለሄኒ ፡ የዐርግ ፡ ወእምህየኒ ፡ ማይ ፡ ወጠል ።} & \\
\textamh{{\Large 29}\ \  ወሖርኩ ፡ ውስተ ፡ መካን ፡ ካልእ ፡ እምነ ፡ መድበራ ፡ መንገለ ፡ ጽባሑ ፡ ለውእቱ ፡ ደብር ፡ ቀረብኩ ። ወበህየ ፡ ርኢኩ ፡ ዕፀወ ፡ ኵነኔ ፡ ፈድፋደ ፡ ቈስቈሰ ፡ መዐዛ ፡ ለስኂን ፡ ወከርቤ ፡ ወዕፀወኒ ፡ ኢይትማሰሉ ።} & \\
\textamh{{\Large 30}\ \  ወላዕሌሁ ፡ ላዕለ ፡ እላንቱ ፡ ላዕለ ፡ ደብረ ፡ ጽባሕ ፡ ወአኮ ፡ ርኁቅ ፡ ወርኢኩ ፡ መካነ ፡ ካልአ ፡ ቈላተ ፡ ማይ ፡ ከመ ፡ ዘኢይትዌዳዕ ፡ ወርኢኩ ፡ ዕፀ ፡ ሠናየ ፡ ወመዐዛሁ ፡ ከመ ፡ ዘሰኪኖን ። ወመንገለ ፡ ክነፊሆሙ ፡ ለቈላት ፡ እሎንቱ ፡ ርኢኩ ፡ ቀናንሞስ ፡ ዘመዐዛ ፡ ወዲበ ፡ እልክቱ ፡ ቀረብኩ ፡ ዘመንገለ ፡ ጽባሕ ።} & \\
\textamh{{\Large 31}\ \  ወርኢኩ ፡ ካልአ ፡ ደብረ ፡ ዘቦቱ ፡ ዕፀው ፡ ወይወፅእ ፡ ማይ ፡ ወይወፅእ ፡ እምኔሁ ፡ ከመ ፡ ኔቄጥሮ ፡ ዘስሙ ፡ ሳሪራ ፡ ወከልበኔን ። ወዲበ ፡ ውእቱ ፡ ደብር ፡ ርኢኩ ፡ ደብረ ፡ ካልአ ፡ ወውስቴቱ ፡ ዕፀው ፡ ዘዓልዋ ፡ ወእልኩ ፡ ዕፀው ፡ ምሉአን ፡ ዘከመ ፡ ከርካዕ ፡ ወጽኑዕ ። ወሶበ ፡ ይነሥእዎ ፡ ለውእቱ ፡ ፍሬ ፡ ይኔይስ ፡ እምኵሉ ፡ አፈው ።} & \\
\textamh{{\Large 32}\ \ ወእምድኅረ ፡ እሉ ፡ አፈው ፡ ለመስዕ ፡ እንዘ ፡ እኔጽር ፡ መልዕልተ ፡ አድባር ፡ ርኢኩ ፡ ፯አድባረ ፡ ምሉዓነ ፡ ሰንበልት ፡ ቅድወ ፡ ወዕፀወ ፡ መዐዛ ፡ ወቀናንሞን ፡ ወፐፐረ ። ወእምህየ ፡ ሖርኩ ፡ መልዕልተ ፡ ርእሶሙ ፡ ለእልኩ ፡ አድባ ፡ ር ፡ እንዘ ፡ ርኁቅ ፡ ውእቱ ፡ ለጽባሕ ፡ ወኀለፍኩ ፡ እንተ ፡ ዲበ ፡ ባሕረ ፡ ኤርትራ ፡ ወእምኔሁ ፡ ርኁቀ ፡ ኮንኩ ፡ ወኀለፍኩ ፡ መልዕልቶ ፡ ለመልአክ ፡ ዙጥኤል ። ወመጻእኩ ፡ ውስተ ፡ ገነተ ፡ ጽድቅ ፡ ወርኢኩ ፡ ካሐካሐቲሆሙ ፡ ለእልክቱ ፡ ዕፀው ፡ ዕፀወ ፡ ብዙኃነ ፡ ወዐቢያነ ፡ ይበቍሉ ፡ በህየ ፡ ወእንዘ ፡ ፄናሆሙ ፡ ሠናይ ፡ ዐቢያን ፡ ወስኖሙ ፡ ብዙኅ ፡ ወስቡሓን ፡ ወዕፀ ፡ ጥበብ ፡ ዘእምኔሁ ፡ በሊዖሙ ፡ የአምርዋ ፡ ለጥበብ ፡ ዐባይ ። ወይመስል ፡ ሐመረ ፡ ጽራእ ፡ ወፍሬሁ ፡ ከመ ፡ አስካለ ፡ ወይን ፡ ጥቀ ፡ ሠናይ ፡ ወፄናሁ ፡ ለውእቱ ፡ ዕፅ ፡ የሐውር ፡ ወይበጽሕ ፡ ነዊኃ ። ወእቤ ፡ ሠናይ ፡ ዝዕፅ ፡ ወከመ ፡ ሠናይ ፡ ወፍሡሕ ፡ ርእየቱ ። ወአውሥአኒ ፡ መልአክ ፡ ቅዱስ ፡ ሩፋኤል ፡ ዘምስሌየ ፡ ሀሎ ፡ ወይቤለኒ ፡ ዝውእቱ ፡ ዕፀ ፡ ጥበብ ፡ ዘእምኔሁ ፡ በልዑ ፡ አቡከ ፡ አረጋዊ ፡ ወእምከ ፡ እቤራዊት ፡ እለ ፡ ቀደሙከ ፡ ወአእመርዋ ፡ ለጥበብ ፡ ወተፈትሐ ፡ አዕይንቲሆሙ ፡ ወአእመሩ ፡ ከመ ፡ ዕራቃኒሆሙ ፡ ሀለዉ ፡ ወተሰዱ ፡ እምገነት ።} & \\
\textamh{{\Large 33}\ \  ወእምህየ ፡ ሖርኩ ፡ እስከ ፡ አጽናፈ ፡ ምድር ፡ ወርኢኩ ፡ በህየ ፡ አራዊተ ፡ ዐበይተ ፡ ወይትዌለጥ ፡ ፩እምካልኡ ፤ ወአዕዋፍሂ ፡ ይትዌለጥ ፡ ገጾም ፡ ወስኖሙ ፡ ወቃሎሙሂ ፡ ይትዌለጥ ፡ ፩እምካልኡ ። ወበጽባሖሙ ፡ ለእሉ ፡ አራዊት ፡ ርኢኩ ፡ አጽናፈ ፡ ምድር ፡ በኀበ ፡ ሰማይ ፡ ያዐርፍ ፡ ወኀዋኅው ፡ ሰማይ ፡ ርኋተ ። ወርኢኩ ፡ እፎ ፡ ይወፅኡ ፡ ከዋክብተ ፡ ሰማይ ፡ ወኈለቁ ፡ ዘእምነ ፡ ይወፅኡ ፡ ኀዋኅው ፡ ወጸሐፍኩ ፡ ኵሎ ፡ ሙጻኦሙ ፡ ለለ፩በኍልቆሙ ፤ ወአስማቲሆሙ ፤ በደርጎሙ ፤ ወምንባሮሙ ፤ ወጊዜሆሙ ፤ ወአውራኂሆሙ ፤ በከመ ፡ አርአየኒ ፡ መልአክ ፡ ኡርኤል ፡ ዘምስሌየ ፡ ሀሎ ። ወኵሎ ፡ አርአየኒ ፡ ሊተ ፡ ወጸሐፎ ፤ ወዓዲ ፡ አስማቲሆሙ ፡ ጸሐፈ ፡ ሊተ ፡ ወትእዛዛቲሆሙ ፡ ወምግባራቲሆሙ ።} & \\
\textamh{{\Large 34}\ \  ወእምህየ ፡ ሖርኩ ፡ መንገለ ፡ መስዕ ፡ በአጽናፈ ፡ ምድር ፡ ወበህየ ፡ ርኢኩ ፡ መንክረ ፡ ዐቢየ ፡ ወስቡሐ ፡ በአጽናፊሃ ፡ ለኵላ ፡ ምድር ። ወበህየ ፡ ርኢኩ ፡ ኀዋኅው ፡ ሰማይ ፡ ፍቱሐተ ፡ በሰማይ ፡ ፫በበ፩እምኔሆሙ ፡ ይወጽኡ ፡ ነፋሳት ፡ በመንገለ ፡ መስዕ ፡ ሶበ ፡ ይነፍሕ ፡ ቍር ፡ ወበረድ ፡ ወአስሐትያ ፡ ወሐመዳ ፡ ወጠል ፡ ወዝናም ። ወእምአሐቲ ፡ ኆኅት ፡ በሠናይ ፡ ይነፍኅ ፡ ወሶበ ፡ በ፪ሆሙ ፡ ኀዋኅው ፡ ይነፍኁ ፡ በኀይል ፡ ወበፃዕር ፡ ይከውን ፡ ዲበ ፡ ምድር ፡ ወበኀይል ፡ ይነፍኁ ።} & \\
\textamh{{\Large 35}\ \ ወእምህየ ፡ ሖርኩ ፡ መንገለ ፡ ዐረብ ፡ በአጽናፈ ፡ ምድር ፡ ወርኢኩ ፡ በህየ ፡ ፫ኀዋኅው ፡ ርኋተ ፡ በከመ ፡ ርኢኩ ፡ በምሥራቅ ፡ በአምጣነ ፡ ኀዋኅው ፡ ወበአምጣነ ፡ ሙፃኣቱ ።} & \\
\textamh{{\Large 36}\ \ ወእምህየ ፡ ሖርኩ ፡ መንገለ ፡ አዜብ ፡ በአጽናፈ ፡ ምድር ፡ ወበህየ ፡ ርኢኩ ፡ ፫ኀዋኅው ፡ ሰማይ ፡ ርኋተ ፡ ወይወፅእ ፡ እምህየ ፡ አዜብ ፡ ወጠል ፡ ወዝናም ፡ ወነፋስ ። ወእምህየ ፡ ሖርኩ ፡ መንገለ ፡ ጽባሕ ፡ በአጽናፈ ፡ ሰማይ ፡ ወበህየ ፡ ርኢኩ ፡ ፫ኀዋኅው ፡ ሰማይ ፡ ርኋተ ፡ መንገለ ፡ ጽባሕ ፡ ወላዕሌሆሙ ፡ ኀዋኅው ፡ ንዑሳን ። በበ ፡ ፩እምእልኩ ፡ ኀዋኅው ፡ ንዑሳን ፡ የኀልፉ ፡ ከዋክብተ ፡ ሰማይ ፡ ወየሐውሩ ፡ ምዕራበ ፡ በፍኖት ፡ እንተ ፡ ተርእየት ፡ ሎሙ ። ወሶበ ፡ ርኢኩ ፡ ባረኩ ፡ ወበኵሉ ፡ ጊዜ ፡ እባርኮ ፡ ለእግዚአ ፡ ስብሐት ፡ ዘገብረ ፡ ተአምራተ ፡ ዐቢያነ ፡ ወስቡሓነ ፡ ከመ ፡ ያርኢ ፡ ዕበየ ፡ ግብሩ ፡ ለመላእክቲሁ ፡ ወለነፍሳተ ፡ ሰብእ ፡ ከመ ፡ ይሰብሑ ፡ ግብሮ ፡ ወኵሉ ፡ ተግባሩ ፡ ከመ ፡ ይርአዩ ፡ ግብረ ፡ ኀይሉ ፡ ወይሰብሕዎ ፡ ለግብረ ፡ እደዊሁ ፡ ዐቢይ ፡ ወይባርክዎ ፡ እስከ ፡ ለዓለም ።} & \\
\textamh{{\Large 37}\ \ ራእይ ፡ ዘርእየ ፡ ካልአ ፡ ራእየ ፡ ጥበብ ፡ ዘርእየ ፡ ሄኖክ ፡ ወልደ ፡ ያሬድ ፡ ወልደ ፡ መላልኤል ፡ ወልደ ፡ ቃይናን ፡ ወልደ ፡ ሄኖስ ፡ ወልደ ፡ ሴት ፡ ወልደ ፡ አዳም ። ወዝርእሱ ፡ ለነገረ ፡ ጥበብ ፡ ዘአንሣእኩ ፡ እትናገር ፡ ወእብል ፡ ለእለ ፡ የኀድሩ ፡ ዲበ ፡ የብስ ፡ ስምዑ ፡ ቀደምት ፡ ወርእዩ ፡ ደኃርያን ፡ ነገረ ፡ ቅዱሰ ፡ እለ ፡ እነግር ፡ ቅድመ ፡ እግዚአ ፡ መናፍስት ። እሉ ፡ ቀዳሚ ፡ ይኀይስ ፡ ብሂል ፡ ወደኃርያንሂ ፡ ኢንከልእ ፡ ርእሳ ፡ ለጥበብ ። እስከ ፡ ይእዜ ፡ ኢተውህበኒ ፡ እምቅድመ ፡ እግዚአ ፡ መናፍስት ፡ ዘነሣእኩ ፡ ጥበበ ፡ በከመ ፡ ሀለይኩ ፡ በከመ ፡ ፈቀደ ፡ እግዚአ ፡ መናፍስት ፡ ዘተውህበኒ ፡ እምኔሁ ፡ ክፍለ ፡ ሕይወት ፡ ዘለዓለም ። ወኮኑ ፡ ብየ ፡ ፫ምሳሌ ፡ ወአነ ፡ አንሣእኩ ፡ እንዘ ፡ እብሎሙ ፡ ለእለ ፡ የኅድሩ ፡ የብሰ ።} & \\
\textamh{{\Large 38}\ \   ምሳሌ ፡ ቀዳሚ ፡ ሶበ ፡ ያስተርኢ ፡ ማኅበረ ፡ ጻድቃን ፡ ወይትኴነኑ ፡ ኃጥኣን ፡ በኀጢአቶሙ ፡ ወእምገጸ ፡ የብስ ፡ ይትሀወኩ ። ወሶበ ፡ ያስተርኢ ፡ ጻድቅ ፡ በገጾሙ ፡ ለጻድቃን ፡ እለ ፡ ኅሩያን ፡ ተግባሮሙ ፡ ስቁል ፡ በእግዚአ ፡ መናፍስት ፡ ወያስተርኢ ፡ ብርሃን ፡ ለጻድቃን ፡ ወለኅሩያን ፡ እለ ፡ የኀድሩ ፡ ዲበ ፡ የብስ ፡ አይቴ ፡ ማኅደረ ፡ ኃጥኣን ፡ ወአይቴ ፡ ምዕራፎሙ ፡ ለእለ ፡ ክህድዎ ፡ ለእግዚአ ፡ መናፍስት ፡ እምኀየሶሙ ፡ ሶበ ፡ ኢተወልዱ ። ወሶበ ፡ ይትከሠታ ፡ ኅቡኣቲሆሙ ፡ ለጻድቃን ፡ ይትኴነኑ ፡ ኃጥኣን ፡ ወይትሀወኩ ፡ ረሲዓን ፡ እምገጸ ፡ ጻድቃን ፡ ወኅሩያን ። ወእምይእዜ ፡ ኢይከውኑ ፡ አዚዛነ ፡ ወኢልዑላነ ፡ እለ ፡ ይእኅዝዋ ፡ ለምድር ፡ ወኢይክሉ ፡ ርእየ ፡ ገጸ ፡ ቅዱሳን ፡ እስመ ፡ ለእግዚአ ፡ መናፍስት ፡ ተርእየ ፡ ብርሃኑ ፡ ለገጸ ፡ ቅዱሳን ፡ ወጻድቃን ፡ ወኅሩያን ። ወነገሥት ፡ አዚዛን ፡ በውእቱ ፡ ጊዜ ፡ ይትኃጐሉ ፡ ወይትወሀቡ ፡ ውስተ ፡ እደ ፡ ጻድቃን ፡ ወቅዱሳን ። ወእምህየ ፡ አልቦ ፡ ዘያስተምህር ፡ ኀበ ፡ እግዚአ ፡ መናፍስት ፡ እስመ ፡ ተወድአ ፡ እንቲአሆሙ ፡ ሕይወት ።} & \\
\textamh{{\Large 39}\ \  ወይከውን ፡ በዝንቱ ፡ መዋዕል ፡ ይወርዱ ፡ ደቂቅ ፡ ኅሩያን ፡ ወቅዱሳን ፡ እምልዑላን ፡ ሰማያት ፡ ወ፩ ፡ ይከውን ፡ ዘርኦሙ ፡ ምስለ ፡ ውሉደ ፡ ሰብእ ። በውእቱ ፡ መዋዕል ፡ ነሥአ ፡ ሄኖክ ፡ መጻሕፍተ ፡ ቅንዐት ፡ ወመዐት ፡ ወመጻሕፍተ ፡ ጕጕዓ ፡ ወሀውክ ፡ ወምሕረት ፡ ኢይከውን ፡ ላዕሌሆሙ ፡ ይቤ ፡ እግዚአ ፡ መናፍስት ። ወበውእቱ ፡ ጊዜ ፡ መሠጠኒ ፡ ደመና ፡ ወዐውሎ ፡ ነፋስ ፡ እምገጻ ፡ ለምድር ፡ ወአንበረኒ ፡ ውስተ ፡ ጽንፈ ፡ ሰማያት ። ወበህየ ፡ ርኢኩ ፡ ራእየ ፡ ካልአ ፡ ማኅደሪሆሙ ፡ ለጻድቃን ፡ ወምስካባቲሆሙ ፡ ለቅዱሳን ። በህየ ፡ ርእያ ፡ አዕይንትየ ፡ ማኅደሪሆሙ ፡ ምስለ ፡ መላእክት ፡ ወምስካባቲሆሙ ፡ ምስለ ፡ ቅዱሳን ፡ ወይስእሉ ፡ ወያስተበቍዑ ፡ ወይጼልዩ ፡ በእንተ ፡ ውሉደ ፡ ሰብእ ፡ ወጽድቅ ፡ ከመ ፡ ማይ ፡ ይውሕዝ ፡ በቅድሜሆሙ ፡ ወምሕረት ፡ ከመ ፡ ጠል ፡ ውስተ ፡ ምድር ፤ ከመዝ ፡ ውእቱ ፡ ማእከሎሙ ፡ ለዓለመ ፡ ዓለም ። ወበውእቱ ፡ መዋዕል ፡ ርእያ ፡ አዕይንትየ ፡ መካነ ፡ ኅሩያን ፡ ዘጽድቅ ፡ ወዘሃይማኖት ፡ ወጽድቅ ፡ ይከውን ፡ በመዋዕሊሆሙ ፡ ወጻድቃን ፡ ወኅሩያን ፡ ኍልቍ ፡ አልቦሙ ፡ ቅድሜሁ ፡ ለዓለመ ፡ ዓለም ። ወርኢኩ ፡ ማኅደሪሆሙ ፡ መትሕተ ፡ አክናፈ ፡ እግዚአ ፡ መናፍስት ፡ ወኵሎሙ ፡ ጻድቃን ፡ ወኅሩያን ፡ በቅድሜሁ ፡ ይትለሐዩ ፡ ከመ ፡ ብርሃነ ፡ እሳት ፡ ወአፉሆሙ ፡ ይመልእ ፡ በረከተ ፡ ወከናፍሪሆሙ ፡ ይሴብሑ ፡ ስሞ ፡ ለእግዚአ ፡ መናፍስት ፡ ወጽድቅ ፡ ቅድሜሁ ፡ ኢየኀልቅ ። ህየ ፡ ፈቀድኩ ፡ እኅድር ፡ ወፈተወቶ ፡ ነፍስየ ፡ ለውእቱ ፡ ማኅደር ፡ በህየ ፡ ኮነ ፡ ክፍልየ ፡ ቅድመ ፡ እስመ ፡ ከመዝ ፡ ጸንዐ ፡ በእንቲአየ ፡ በቅድመ ፡ እግዚአ ፡ መናፍስት ። ወበውእቶን ፡ መዋዕል ፡ ሰባሕኩ ፡ ወአልዓልኩ ፡ ስሞ ፡ ለእግዚአ ፡ መናፍስት ፡ በረከተ ፡ ወስብሐተ ፡ እስመ ፡ ውእቱ ፡ አጽንዐኒ ፡ በበረከት ፡ ወስብሐት ፡ በከመ ፡ ፈቃዱ ፡ ለእግዚአ ፡ መናፍስት ። ወጕንዱየ ፡ ርእያ ፡ አዕይንትየ ፡ በውእቱ ፡ መካን ፡ ወባረክዎ ፡ እንዘ ፡ እብል ፡ ቡሩክ ፡ ውእቱ ፡ ወይትባረክ ፡ እምቅድም ፡ ወእስከ ፡ ለዓለም ። ወበቅድሜሁ ፡ አቦ ፡ ማኅለቅት ፡ ውእቱ ፡ ያአምር ፡ ዘእንበለ ፡ ይትፈጠር ፡ ዓለም ፡ ምንት ፡ ውእቱ ፡ ዓለም ፡ ወለትውልደ ፡ ትውልድ ፡ ዘይከውን ። ይባርኩከ ፡ እለ ፡ ኢይነውሙ ፡ ወይቀውሙ ፡ በቅድመ ፡ ስብሐቲከ ፡ ወይባርኩከ ፡ ወይሴብሑ ፡ ወያሌዕሉ ፡ እንዘ ፡ ይብሉ ፡ ቅዱስ ፡ ቅዱስ ፡ ቅዱስ ፡ እግዚአ ፡ መናፍስት ፡ ይመልእ ፡ ምድረ ፡ መንፈሳት ። ወበህየ ፡ ርእያ ፡ አዕይንትየ ፡ ኵሎ ፡ እለ ፡ ኢይነውሙ ፡ ይቀውሙ ፡ ቅድሜሁ ፡ ወይባርኩ ፡ ወይብሉ ፡ ቡሩክ ፡ አንተ ፡ ወቡሩክ ፡ ስሙ ፡ ለእግዚእ ፡ ለዓለመ ፡ ዓለም ። ወተወለጠ ፡ ገጽየ ፡ እስከ ፡ ስእንኩ ፡ ነጽሮ ።} & \\
\textamh{{\Large 40}\ \  ወርኢኩ ፡ እምድኅረዝ ፡ አእላፈ ፡ አእላፍት ፡ ወትእልፊተ ፡ ትእልፊት ፡ ወአልቦሙ ፡ ኍልቍ ፡ ወሐሳብ ፡ እለ ፡ ይቀውሙ ፡ ቅድመ ፡ ስብሐተ ፡ እግዚአ ፡ መናፍስት ። ርኢኩ ፡ ወበ፬ክነፊሁ ፡ ለእግዚአ ፡ መናፍስት ፡ ርኢኢኩ ፡ ፬ገጸ ፡ ካልአ ፡ እምእለ ፡ ይቀውሙ ፡ ወአስማቲሆሙ ፡ አእመርኩ ፡ ዘአይድዐኒ ፡ አስማቲሆሙ ፡ መልአክ ፡ ዘመጽአ ፡ ምስሌየ ፡ ወኵሎ ፡ ኅቡኣተ ፡ አርአየኒ ። ወሰማዕኩ ፡ ቃሎሙ ፡ ለእልኩ ፡ ፬ገጽ ፡ እንዘ ፡ ይሴብሑ ፡ ቅድመ ፡ እግዚአ ፡ ስብሐት ። ቃል ፡ ቀዳማዊ ፡ ይባርኮ ፡ ለእግዚአ ፡ መናፍስት ፡ ለዓለመ ፡ ዓለም ። ወቃለ ፡ ካልአ ፡ ሰማዕኩ ፡ እንዘ ፡ ይባርኮ ፡ ለኅሩይ ፡ ወለኅሩያን ፡ እለ ፡ ስቁላን ፡ በእግዚአ ፡ መናፍስት ። ወሣልሰ ፡ ቃለ ፡ ሰማዕኩ ፡ እንዘ ፡ ይስእሉ ፡ ወይጼልዩ ፡ በእንተ ፡ እለ ፡ የኀድሩ ፡ ውስተ ፡ የብስ ፡ ወያስተበቍዑ ፡ በስሙ ፡ ለእግዚአ ፡ መናፍስት ። ወቃለ ፡ ራብዐ ፡ ሰማዕኩ ፡ እንዘ ፡ ይሰድዶሙ ፡ ለሰይጣናት ፡ ወኢየኀድጎሙ ፡ ይባኡ ፡ ኀበ ፡ እግዚአ ፡ መናፍስት ፡ ከመ ፡ ያስተዋድይዎሙ ፡ ለእለ ፡ የኀድሩ ፡ ዲበ ፡ የብስ ። ወእምድኅረዝ ፡ ተስእልክዎ ፡ ለመልአከ ፡ ሰላም ፡ ዘየሐውር ፡ ምስሌየ ፡ ዘውእቱ ፡ አርአየኒ ፡ ኵሎ ፡ ዘኅቡእ ፡ ወእቤሎ ፡ መኑ ፡ ውእቶሙ ፡ እሉ ፡ ፬ገጽ ፡ ዘርኢኩ ፡ ወእለ ፡ ሰማዕኩ ፡ ቃሎሙ ፡ ወጸሐፍክዎሙ ። ወይቤለኒ ፡ ዝቀዳማዊ ፡ ውእቱ ፡ መሐሪ ፡ ወርኁቀ ፡ መዓት ፡ ቅዱስ ፡ ሚካኤል ፤ ወካልእ ፡ ዘዲበ ፡ ኵሉ ፡ ሕማም ፡ ወዲበ ፡ ኵሉ ፡ ቍስል ፡ ዘውሉደ ፡ ሰብእ ፡ ውእቱ ፡ ሩፋኤል ፤ ወሣልስ ፡ ዘዲበ ፡ ኵሉ ፡ ኃይል ፡ ውእቱ ፡ ቅዱስ ፡ ገብርኤል ፤ ወራብዕ ፡ ዘዲበ ፡ ንስሓ ፡ ወለተስፋ ፡ እለ ፡ ይወርሱ ፡ ሕይወተ ፡ ዘለዓለም ፡ ውእቱ ፡ ፋኑኤል ። ወእሉ ፡ ፬መላእክቲሁ ፡ ለእግዚአብሔር ፡ ልዑል ፡ ወ፬ቃለ ፡ ሰማዕኩ ፡ በውእቶን ፡ መዋዕል ።} & \\
\textamh{{\Large 41}\ \ ወእምድኅረዝ ፡ ርኢኩ ፡ ኵሎ ፡ ኅብኣቲሆሙ ፡ ለሰማያት ፡ ወመንግሥት ፡ እፎ ፡ ትትከፈል ፡ ወተግባረ ፡ ሰብእ ፡ ከመ ፡ በመዳልው ፡ ይደለዉ ። በህየ ፡ ርኢኩ ፡ ማኅደሪሆሙ ፡ ለኅሩያን ፡ ወማኅደሪሆሙ ፡ ለቅዱሳን ፡ ወርእያ ፡ አዕይንትየ ፡ በህየ ፡ ኵሎሙ ፡ ኃጥኣን ፡ እንዘ ፡ ይሰደዱ ፡ እምህየ ፡ እለ ፡ ይክህድዎ ፡ ለስመ ፡ እግዚአ ፡ መናፍስት ፡ ወይስሕብዎሙ ፡ ወቀዊም ፡ አልቦሙ ፡ በመቅሠፍት ፡ እንተ ፡ ትወጽእ ፡ እምእግዚአ ፡ መናፍስት ። ወበህየ ፡ ርእያ ፡ አዕይንትየ ፡ ኅቡአተ ፡ መባርቅት ፡ ወነጐድጓድ ፡ ወኅቡአተ ፡ ነፋሳት ፡ እፎ ፡ ይትከፈሉ ፡ ከመ ፡ ይንፍሑ ፡ ዲበ ፡ ምድር ፡ ወኅቡአተ ፡ ደመናት ፡ ወጠል ፡ ወበህየ ፡ ርኢኩ ፡ እምኀበ ፡ ይወጽእ ፡ በውእቱ ፡ መካን ፡ ወእምህየ ፡ ይጸግቡ ፡ ፀበለ ፡ ምድር ። ወበህየ ፡ ርኢኩ ፡ መዛግብተ ፡ ዕፅዋነ ፡ ወእምኔሆሙ ፡ ይትከፈሉ ፡ ነፋሳት ፡ ወመዝገበ ፡ በረድ ፡ ወመዝገበ ፡ ጊሜ ፡ ወዘደመናት ፡ ወደመና ፡ ዚአሁ ፡ እምላዕለ ፡ ምድር ፡ የኅድር ፡ እምቅድመ ፡ ዓለም ። ወርኢኩ ፡ መዛግብተ ፡ ፀሐይ ፡ ወዘወርኅ ፡ እምአይቴ ፡ ይወጽኡ ፡ ወአይቴ ፡ ይገብኡ ፡ ወግብአቶሙ ፡ ስብሕ ፤ ወእፎ ፡ ይከብር ፡ ፩እምካልኡ ፡ ወምኋሮሙ ፡ ብዑል ፡ ወኢየኀልፉ ፡ ምኋረ ፡ ወኢይዌስኩ ፡ ወኢየሐፅፁ ፡ እምኋረ ፡ ዚአሆሙ ፡ ወሃይማኖቶሙ ፡ የዐቅቡ ፡ ፩ ፡ ምስለ ፡ ካልኡ ፡ በመሐላ ፡ ዘነበሩ ። ወይወፅእ ፡ ቅድመ ፡ ፀሐይ ፡ ወይገብር ፡ ፍኖቶ ፡ በትእዛዘ ፡ እግዚአ ፡ መናፍስት ፡ ወይጸንዕ ፡ ስሙ ፡ ለዓለመ ፡ ዓለም ። ወእምድኅረዝ ፡ ፍኖት ፡ ዘኅቡእ ፡ ወዘክሡት ፡ ዘወርኅ ፡ ወምኋረ ፡ ፍኖቱ ፡ ይፌጽም ፡ በውእቱ ፡ መካን ፡ በመዓልት ፡ ወበሌሊት ፡ ፩ ፡ ለካልኡ ፡ ይኔጽር ፡ በቅድመ ፡ እግዚአ ፡ መናፍስት ፡ ወየአኵቱ ፡ ወይሴብሑ ፡ ወኢያዐርፉ ፡ እስመ ፡ አኰቴቶሙ ፡ ዕረፍት ፡ ውእቱ ፡ ሎሙ ። እስመ ፡ ለፀሐይ ፡ ብሩህ ፡ ብዙኅ ፡ ምያጤ ፡ ቦቱ ፡ ለበረከት ፡ ወለመርገም ፡ ወምርዋጸ ፡ ፍኖቱ ፡ ለወርኅ ፡ ለጻድቃን ፡ ብርሃን ፡ ወለኃጥኣን ፡ ጽልመት ፡ በስሙ ፡ ለእግዚእ ፡ ዘፈጠረ ፡ ማእከለ ፡ ብርሃን ፡ ወማእከለ ፡ ጽልመት ፡ ወከፈለ ፡ መንፈሶሙ ፡ ለሰብእ ፡ ወአጽንዐ ፡ መንፈሶሙ ፡ ለጻድቃን ፡ በስመ ፡ ጽድቀ ፡ ዚአሁ ። እስመ ፡ መልአክ ፡ ኢይከልእ ፡ ወሥልጣን ፡ ኢይክል ፡ ከሊአ ፡ እስመ ፡ መኰንን ፡ ለኵሎሙ ፡ ይሬኢ ፡ ወእሎንተ ፡ ኵሎሙ ፡ በቅድሜሁ ፡ ውእቱ ፡ ይኴንን ።} & \\
\textamh{{\Large 42}\ \  ጥበብ ፡ መካነ ፡ ኢረከበት ፡ ኀበ ፡ ተኀድር ፡ ወሀለወት ፡ ማኅደራ ፡ ውስተ ፡ ሰማያት ። መጽአት ፡ ጥበብ ፡ ከመ ፡ ትኅድር ፡ ውስተ ፡ ውሉደ ፡ ሰብእ ፡ ወኢረከበት ፡ ማኅደረ ፤ ጥበብ ፡ ውስተ ፡ መካና ፡ ገብአት ፡ ወተፅዕነት ፡ ማእከለ ፡ መላእክት ። ወዐመፃ ፡ ወፅአት ፡ እመዛግብቲሃ ፡ ዘኢፈቀደት ፡ ረከበት ፡ ወኀደረት ፡ ውስቴቶሙ ፡ ከመ ፡ ዝናም ፡ በበድው ፡ ወከመ ፡ ጠል ፡ በምድር ፡ ጽምእት ።} & \\
\textamh{{\Large 43}\ \  ወርኢኩ ፡ ካልአ ፡ መባርቅተ ፡ ወከዋክብተ ፡ ሰማይ ፡ ወርኢኩ ፡ ከመ ፡ ይጼውዖሙ ፡ ለኵሎሙ ፡ በበአስማቲሆሙ ፡ ወይሰምዕዎ ። ወርኢክዎ ፡ በመዳልወ ፡ ጽድቅ ፡ ከመ ፡ ይደለዉ ፡ በብርሃናቲሆሙ ፡ በራኅበ ፡ መካናቲሆሙ ፡ ወዕለተ ፡ ኩነቶሙ ፡ ወሚጠቶሙ ፤ መብረቅ ፡ መብረቀ ፡ ይወልድ ፤ ወሚጠቶሙ ፡ በኍልቈ ፡ መላእክት ፡ ወሃይማኖቶሙ ፡ የዐቅቡ ፡ በበይናቲሆሙ ። ወተስእልክዎ ፡ ለመልአክ ፡ ዘየሐውር ፡ ምስሌየ ፡ ዘአርአየኒ ፡ ዘኅቡእ ፡ ምንት ፡ እሎንቱ ። ወይቤለኒ ፡ ምስለ ፡ ዘዚአሆሙ ፡ አርአየከ ፡ እግዚአ ፡ መናፍስት ፡ እሉ ፡ እሙንቱ ፡ አስማቲሆሙ ፡ ለጻድቃን ፡ እለ ፡ የኅድሩ ፡ ዲበ ፡ የብስ ፡ ወየአምኑ ፡ በስሙ ፡ ለእግዚአ ፡ መናፍስት ፡ ለዓለመ ፡ ዓለም ።} & \\
\textamh{{\Large 44}\ \ ወካልኣተ ፡ ርኢኩ ፡ በእንተ ፡ መብረቅ ፡ እፎ ፡ ይቀውሙ ፡ እምከዋክብት ፡ ወይከውኑ ፡ መብረቀ ፡ ወኢይክሉ ፡ ኀዲገ ፡ ምስሌሆሙ ።} & \\
\textamh{{\Large 45}\ \ ወዝካልእ ፡ ምሳሌ ፡ ዲበ ፡ እለ ፡ ይክህዱ ፡ ስሞ ፡ ለማኅደረ ፡ ቅዱሳን ፡ ወለእግዚአ ፡ መናፍስት ። እሰማየ ፡ የዐርጉ ፡ ወኢምድረ ፡ ይበጽሑ ፡ ከመዝ ፡ ይከውን ፡ ክፍለ ፡ ኃጥኣን ፡ እለ ፡ ይክሕዱ ፡ ስሞ ፡ ለእግዚአ ፡ መናፍስት ፡ እለ ፡ ከመዝ ፡ ይትዐቀብ ፡ ለዕለተ ፡ ሥራኅ ፡ ወምንዳቤ ። በይእቲ ፡ ዕለት ፡ ይነብር ፡ በመንበረ ፡ ስብሐት ፡ ኅሩይ ፡ ወየኀሪ ፡ ምግባሪሆሙ ፡ ወምዕራፎሙ ፡ ኍልቍ ፡ አልቦሙ ፡ ወመንፈሶሙ ፡ በማእከሎሙ ፡ ትጸንዕ ፡ ሶበ ፡ ርእይዎ ፡ ለኅሩየ ፡ ዚአየ ፡ ወለእለ ፡ ሰከዩ ፡ ስምየ ፡ ቅዱሰ ፡ ወስቡሐ ። ወበይእቲ ፡ ዕለት ፡ አነብሮ ፡ በማእከሎሙ ፡ ለኅሩየ ፡ ዚአየ ፡ ወእዌልጣ ፡ ለሰማይ ፡ ወእገብራ ፡ ለበረከት ፡ ወብርሃን ፡ ዘለዓለም ። ወእዌልጣ ፡ ለየብስ ፡ ወእገብራ ፡ ለበረከት ፡ ወለኅሩያነ ፡ ዚአየ ፡ አነብሮሙ ፡ ውስቴታ ፡ ወእለ ፡ ይገብሩ ፡ ኀጢአተ ፡ ወጌጋየ ፡ እይከይዱ ፡ ውስቴታ ። እስመ ፡ አነ ፡ ርኢክዎሙ ፡ ወአጽገብክዎሙ ፡ በሰላም ፡ ለጻድቃንየ ፡ ወአንበርክዎሙ ፡ ቅድሜየ ፡ ወቀርበት ፡ ኀቤየ ፡ ኵነኔ ፡ ለኃጥኣን ፡ ከመ ፡ ኣህጕሎሙ ፡ እምገጸ ፡ ምድር ።} & \\
\textamh{{\Large 46}\ \ ወበህየ ፡ ርኢኩ ፡ ዘሎቱ ፡ ርእሰ ፡ መዋዕል ፡ ወርእሱ ፡ ከመ ፡ ፀምር ፡ ጸዐዳ ፡ ወምስሌሁ ፡ ካልእ ፡ ዘገጹ ፡ ከመ ፡ ርእየተ ፡ ሰብእ ፡ ወምሉእ ፡ ጸጋ ፡ ገጹ ፡ ከመ ፡ ፩እመላእክት ፡ ቅዱሳን ። ወተስእልክዎ ፡ ለ፩እመላእክት ፡ ዘየሐውር ፡ ምስሌየ ፡ ወኵሎ ፡ ኅቡኣተ ፡ ዘአርአየኒ ፡ በእንተ ፡ ዝኩ ፡ ወልደ ፡ ሰብእ ፡ መኑ ፡ ውእቱ ፡ ወእምአይቴ ፡ ይከውን ፡ ውእቱ ፡ በእንተ ፡ ምንት ፡ ምስለ ፡ ርእሰ ፡ መዋዕል ፡ የሐውር ። ወአውሥአኒ ፡ ወይቤለኒ ፡ ዝንቱ ፡ ውእቱ ፡ ወልደ ፡ ሰብእ ፡ ዘሎቱ ፡ ኮነ ፡ ጽድቅ ፡ ወጽድቅ ፡ ምስሌሁ ፡ ኃደረ ፡ ወኵሎ ፡ መዛግብተ ፡ ዘኅቡእ ፡ ውእቱ ፡ ይከሥት ፡ እስመ ፡ እግዚአ ፡ መናፍስት ፡ ኪያሁ ፡ ኀረየ ፡ ወዘክፍሉ ፡ ኵሎ ፡ ሞአ ፡ በቅድመ ፡ እግዚአ ፡ መናፍስት ፡ በርትዕ ፡ ለዓለም ። ወዝንቱ ፡ ወልደ ፡ ሰብእ ፡ ዘርኢከ ፡ ያነሥኦሙ ፡ ለነገሥት ፡ ወለኀያላን ፡ እምስካባቲሆሙ ፡ ወለጽኑዓን ፡ እመናብርቲሆሙ ፡ ወይፈትሕ ፡ ልጓማተ ፡ ጽኑዓን ፡ ወያደቅቅ ፡ አስናነ ፡ ኃጥኣን ። ወይገፈትዖሙ ፡ ለነገሥት ፡ እመናብርቲሆሙ ፡ ወእመንግሥቶሙ ፡ እስመ ፡ እያሌዕልዎ ፡ ወኢይሴብሕዎ ፡ ወኢይገንዩ ፡ እምአይቴ ፡ ተውህበት ፡ ሎሙ ፡ መንግሥት ። ወገጸ ፡ ጽኑዓን ፡ ይገፈትዕ ፡ ወይመልኦሙ ፡ ኀፍረት ፡ ወጽልመት ፡ ይከውኖሙ ፡ ማኅደሪሆሙ ፡ ወዕፄያት ፡ ይከውኖሙ ፡ ምስካቦሙ ፡ ወኢይሴፈዉ ፡ ከመ ፡ ይትነሥኡ ፡ እምስካባቲሆሙ ፡ እስመ ፡ ኢያሌዕሉ ፡ ስሞ ፡ ለእግዚአ ፡ መናፍስት ። ወእሙንቱ ፡ ኮኑ ፡ እለ ፡ ይኴንኑ ፡ ከዋክብተ ፡ ሰማይ ፡ ወያሌዕሉ ፡ እደዊሆሙ ፡ ውስተ ፡ ልዑል ፡ ወይከይዱ ፡ ዲበ ፡ የብስ ፡ ወየኀድሩ ፡ ውስቴታ ፡ ወኵሉ ፡ ተግባሮሙ ፡ ዐመፃ ፡ ወያርእዩ ፡ ተግባሮሙ ፡ ዐመፃ ፡ ወኀይሎሙ ፡ ዲበ ፡ ብዕሎሙ ፡ ወሃይማኖቶሙ ፡ ኮነ ፡ ለአማልክት ፡ እለ ፡ ገብሩ ፡ በእደዊሆሙ ፡ ወክሕድዎ ፡ ለስሙ ፡ ለእግዚአ ፡ መናፍስት ። ወይሰደዱ ፡ እምአብያተ ፡ ምስትጉቡአ ፡ ዚአሁ ፡ ወለመሃይምናን ፡ እለ ፡ ስቁላን ፡ በስሙ ፡ ለእግዚአ ፡ መናፍስት ።} & \\
\textamh{{\Large 47}\ \ ወበውእቱ ፡ መዋዕል ፡ ዐርገት ፡ ጸሎተ ፡ ጻድቃን ፡ ወደመ ፡ ጻድቅ ፡ እምነ ፡ ምድር ፡ ቅድመ ፡ እግዚአ ፡ መናፍስት ። በእሉ ፡ መዋዕል ፡ የኀብሩ ፡ ቅዱሳን ፡ እለ ፡ ይነብሩ ፡ መልዕልተ ፡ ሰማያት ፡ በ፩ቃል ፡ ወያስተበቍዑ ፡ ወይጼልዩ ፡ ወይሴብሑ ፡ ወያአኵቱ ፡ ወይባርኩ ፡ ለስሙ ፡ ለእግዚአ ፡ መናፍስት ፡ በእንተ ፡ ደመ ፡ ጻድቃን ፡ ዘተክዕወ ፡ ወጸሎቶሙ ፡ ለጻድቃን ፡ ከመ ፡ ኢትፀራእ ፡ በቅድመ ፡ እግዚአ ፡ መናፍስት ፡ ከመ ፡ ይትገበር ፡ ሎሙ ፡ ኵነኔ ፡ ወትዕግሥት ፡ ኢይኩን ፡ ሎሙ ፡ ለዓለም ። ወበእማንቱ ፡ መዋዕል ፡ ርኢክዎ ፡ ለርእሰ ፡ መዋዕል ፡ ሶበ ፡ ነበረ ፡ በመንበረ ፡ ስብሐቲሁ ፡ ወመጻሕፍተ ፡ ሕያዋን ፡ በቅድሜሁ ፡ ተከሥቱ ፡ ወኵሉ ፡ ኀይሉ ፡ ዘመልዕልተ ፡ ሰማያት ፡ ወዐውደ ፡ ዚአሁ ፡ ይቀውሙ ፡ ቅድሜሁ ። ወልቦሙ ፡ ለቅዱሳን ፡ ትመልእ ፡ ፍሥሐ ፡ እስመ ፡ በጽሐ ፡ ኍልቋ ፡ ለጽድቅ ፡ ወጸሎቶሙ ፡ ለጻድቃን ፡ ተሰምዐ ፡ ወደሙ ፡ ለጻድቅ ፡ በቅድመ ፡ እግዚአ ፡ መናፍስት ፡ ተፈቅደ ።} & \\
\textamh{{\Large 48}\ \ ወበውእቱ ፡ መካን ፡ ርኢኩ ፡ ነቅዐ ፡ ጽድቅ ፡ ወኢይትኌለቍ ፡ ወበዐውዱ ፡ የዐውዶ ፡ ብዙኅ ፡ አንቅዕተ ፡ ጥበብ ፡ ወኵሎሙ ፡ ጽሙኣን ፡ ይሰትዩ ፡ እምኔሆሙ ፡ ወይትመልኡ ፡ ጥበበ ፡ ወማኅደሪሆሙ ፡ ምስለ ፡ ጻድቃን ፡ ወቅዱሳን ፡ ወኅሩያን ። ወበይእቲ ፡ ሰዓት ፡ ተጸውዐ ፡ ዝኩ ፡ ወልደ ፡ ሰብእ ፡ በኀበ ፡ እግዚአ ፡ መናፍስት ፡ ወስሙ ፡ መቅድመ ፡ ርእሰ ፡ መዋዕል ። ወዘእንበለ ፡ ይትፈጠር ፡ ፀሐይ ፡ ወተአምር ፡ ዘእንበለ ፡ ይትገበሩ ፡ ከዋክብተ ፡ ሰማይ ፡ ወስሙ ፡ ተጸውዐ ፡ በቅድመ ፡ እግዚአ ፡ መናፍስት ። ውእቱ ፡ ይከውን ፡ በትረ ፡ ለጻድቃን ፡ ወቅዱሳን ፡ ከመ ፡ ቦቱ ፡ ይትመርጐዙ ፡ ወኢይደቁ ፡ ወውእቱ ፡ ብርሃነ ፡ አሕዛብ ፡ ወውእቱ ፡ ይከውን ፡ ተስፋ ፡ ለእለ ፡ የሐሙ ፡ በልቦሙ ። ይወድቁ ፡ ወይሰግዱ ፡ ቅድሜሁ ፡ ኵሎሙ ፡ እለ ፡ የኀድሩ ፡ ዲበ ፡ የብስ ፡ ወይባርክዎ ፡ ወይሴብሕዎ ፡ ወይዜምሩ ፡ ሎቱ ፡ ለስመ ፡ እግዚአ ፡ መናፍስት ። ወበእንተዝ ፡ ኮነ ፡ ኅሩየ ፡ ወኅቡአ ፡ በቅድሜሁ ፡ እምቅድመ ፡ ይትፈጠር ፡ ዓለም ፡ ወእስከ ፡ ለዓለም ፡ በቅድሜሁ ። ወከሠቶ ፡ ለቅዱሳን ፡ ወለጻድቃን ፡ ጥበቡ ፡ ለእግዚአ ፡ መናፍስት ፡ እስመ ፡ ዐቀበ ፡ ክፍሎሙ ፡ ለጻድቃን ፡ እስመ ፡ ጸልእዎ ፡ ወመነንዎ ፡ ለዝ ፡ ዓለም ፡ ዘዐመፃ ፡ ወኵሎ ፡ ምግባሮ ፡ ወፍናዊሁ ፡ ጸልኡ ፡ በስሙ ፡ ለእግዚአ ፡ መናፍስት ፡ እስመ ፡ በስመ ፡ ዚአሁ ፡ ይድኅኑ ፡ ወፈቃዴ ፡ ኮነ ፡ ለሕይወቶሙ ። ወበውእቱ ፡ መዋዕል ፡ ኮኑ ፡ ትሑታነ ፡ ገጽ ፡ ነገሥተ ፡ ምድር ፡ ወጽኑዓን ፡ እለ ፡ ይእኅዝዋ ፡ ለየብስ ፡ በእንተ ፡ ምግባረ ፡ እደዊሆሙ ፡ እስመ ፡ በዕለተ ፡ ጻዕቆሙ ፡ ወጻሕቦሙ ፡ ኢያድኅኑ ፡ ነፍሶሙ ። ወውስተ ፡ እደዊሆሙ ፡ ለኅሩያነ ፡ ዚአየ ፡ እወድዮሙ ፡ ከመ ፡ ሣዕር ፡ ውስተ ፡ እሳት ፡ ወከመ ፡ ዐረር ፡ ውስተ ፡ ማይ ፡ ከመዝ ፡ ይውዕዩ ፡ እምቅድመ ፡ ገጸ ፡ ጻድቃን ፡ ወይሰጠሙ ፡ እምቅድመ ፡ ገጸ ፡ ቅዱሳን ፡ ወኢይትረከብ ፡ ሎሙ ፡ አሰር ። ወበዕለተ ፡ ጻሕበ ፡ ዚአሆሙ ፡ ዕረፍት ፡ ትከውን ፡ በዲበ ፡ ምድር ፡ ወበቅድሜሁ ፡ ይወድቁ ፡ ወኢይትነሥኡ ፡ ወአልቦ ፡ ዘይትሜጠዎሙ ፡ በእደዊሁ ፡ ወያነሥኦሙ ፡ እስመ ፡ ክሕድዎ ፡ ለእግዚአ ፡ መናፍስት ፡ ወለመሲሑ ፡ ወይትባረክ ፡ ስሙ ፡ ለእግዚአ ፡ መናፍስት ።} & \\
\textamh{{\Large 49}\ \  እስመ ፡ ጥበብ ፡ ተክዕወ ፡ ከመ ፡ ማይ ፡ ወስብሐት ፡ ኢተኀልቅ ፡ ቅድሜሁ ፡ ለዓለመ ፡ ዓለም ። እስመ ፡ ኀያል ፡ ውእቱ ፡ በኵሉ ፡ ኅቡኣተ ፡ ጽድቅ ፡ ወዐመፃ ፡ ከመ ፡ ጽላሎት ፡ የኀልፍ ፡ ወምቅዋም ፡ አልቦ ፡ እስመ ፡ ቆመ ፡ ኅሩይ ፡ በቅድመ ፡ እግዚአ ፡ መናፍስት ፡ ወስብሐቲሁ ፡ ለዓለመ ፡ ዓለም ፡ ወኀይሉ ፡ ለትውልደ ፡ ትውልድ ። ወቦቱ ፡ የኀድር ፡ መንፈሰ ፡ ጥበብ ፡ ወመንፈሰ ፡ ዘያሌቡ ፡ ወመንፈሰ ፡ ትምህርት ፡ ወኀይል ፡ ወመንፈሰ ፡ እለ ፡ ኖሙ ፡ በጽድቅ ። ወውእቱ ፡ ይኴንን ፡ ዘኅቡኣት ፡ ወነገረ ፡ በክ ፡ እልቦ ፡ ዘይክል ፡ ብሂለ ፡ በቅድሜሁ ፡ እስመ ፡ ኅሩይ ፡ ውእቱ ፡ በቅድመ ፡ እግዚአ ፡ መናፍስት ፡ በከመ ፡ ውእቱ ፡ ፈቀደ ።} & \\
\textamh{{\Large 50}\ \ ወበእማንቱ ፡ መዋዕል ፡ ሚጠት ፡ ትከውን ፡ ለቅዱሳን ፡ ወለኅሩያን ፡ ወብርሃነ ፡ መዋዕ ፡ ዲቤሆሙ ፡ የኀድር ፡ ወስብሐት ፡ ወክብር ፡ ለቅዱሳን ፡ ይትመየጥ ። ወበዕለት ፡ እንተ ፡ ጻሕብ ፡ ትዘገብ ፡ እኪት ፡ ላዕለ ፡ ኃጥኣን ፡ ወይመውኡ ፡ ጻድቃን ፡ በስሙ ፡ ለእግዚአ ፡ መናፍስት ፡ ወያርኢ ፡ ለካልኣን ፡ ከመ ፡ ይነስሑ ፡ ወይኅድጉ ፡ ምግባረ ፡ እደዊሆሙ ። ወኢይከውን ፡ ሎሙ ፡ ክብር ፡ በቅድመ ፡ እግዚአ ፡ መናፍስት ፡ ወበስሙ ፡ ይድኅኑ ፡ ወእግዚአ ፡ መናፍስት ፡ ይምሕሮሙ ፡ እስመ ፡ ብዙኅ ፡ ምሕረቱ ። ወጻድቅ ፡ ውእቱ ፡ በኵነኔሁ ፡ ወበቅድመ ፡ ስብሐቲሁ ፡ ወዐመፃ ፡ እትቀውም ፡ በኵነኔሁ ፡ ዘኢይኔስሕ ፡ በቅድሜሁ ፡ ይትሐጐል ። ወእምይእዜሰ ፡ ኢይምሕሮሙ ፡ ይቤ ፡ እግዚአ ፡ መናፍስት ።} & \\
\textamh{{\Large 51}\ \ ወበእማንቱ ፡ መዋዕል ፡ ታገብእ ፡ ምድር ፡ ማኅፀንታ ፡ ወሲኦል ፡ ታገብእ ፡ ማኅፀንታ ፡ ዘተመጠወት ፡ ወኀጕል ፡ ያገብእ ፡ ዘይፈዲ ። ወየኀሪ ፡ ጻድቃነ ፡ ወቅዱሳነ ፡ እምኔሆሙ ፡ እስመ ፡ ቀርበት ፡ ዕለት ፡ ከመ ፡ እሙንቱ ፡ ይድኀኑ ። ወኅሩይ ፡ በእማንቱ ፡ መዋዕል ፡ ዲበ ፡ መንበሩ ፡ ይነብር ፡ ወኵሉ ፡ ኅቡኣተ ፡ ጥበብ ፡ እምሕሊና ፡ አፉሁ ፡ ይወፅእ ፡ እስመ ፡ እግዚአ ፡ መናፍስት ፡ ወሀቦ ፡ ወሰብሖ ። ወበእማንቱ ፡ መዋዕል ፡ ይዘፍኑ ፡ አድባር ፡ ከመ ፡ ሐራጊት ፡ ወአውግርኒ ፡ ያንፈርዕፅ ፡ ከመ ፡ መሐስዕ ፡ ጽጉባነ ፡ ሐሊብ ፡ ወይከውኑ ፡ ኵሎሙ ፡ መላእክተ ፡ በሰማይ ። ገጾሙ ፡ ይበርህ ፡ በፍሥሐ ፡ እስመ ፡ በእማንቱ ፡ መዋዕል ፡ ኅሩይ ፡ ተንሥአ ፡ ወምድር ፡ ትትፌሣሕ ፡ ወጻድቃን ፡ ዲቤሃ ፡ የኀድሩ ፡ ወኅሩያን ፡ ውስቴታ ፡ የሐውሩ ፡ ወያንሶስዉ ።} & \\
\textamh{{\Large 52}\ \ ወእምድኅረ ፡ እማንቱ ፡ መዋዕል ፡ በውእቱ ፡ መካን ፡ ኀበ ፡ ርኢኩ ፡ ኵሎ ፡ ራእያተ ፡ ዘኅቡእ ፡ እስመ ፡ ተመሠጥኩ ፡ በነኰርኳረ ፡ ነፋስ ፡ ወወሰዱኒ ፡ ውስተ ፡ ዐረብ ። በህየ ፡ ርእያ ፡ አዕይንትየ ፡ ኅቡኣተ ፡ ሰማይ ፡ ኵሎ ፡ ዘይከውን ፡ ሀሎ ፡ በዲበ ፡ ምድር ፡ ደብረ ፡ ሐፂን ፤ ወደብረ ፡ ፀሪቅ ፤ ወደብረ ፡ ብሩር ፤ ወደብረ ፡ ወርቅ ፤ ወደብረ ፡ ነጠብጣብ ፤ ወደብረ ፡ ዐረር ። ወተስእልክዎ ፡ ለመልአክ ፡ ዘየሐውር ፡ ምስሌየ ፡ እንዘ ፡ እብል ፡ ምንት ፡ ውእቱ ፡ እሉ ፡ እሙንቱ ፡ እለ ፡ በኅቡዕ ፡ ርኢኩ ። ወይቤለኒ ፡ እሉ ፡ ኵሎሙ ፡ ዘርኢከ ፡ ለሥልጣነ ፡ መሲሑ ፡ እሙንቱ ፡ ይከውኑ ፡ ከመ ፡ የአዝዝ ፡ ወይትኀየል ፡ ዲበ ፡ ምድር ። ወአውሥአኒ ፡ እንዘ ፡ ይብል ፡ ውእቱ ፡ መልአከ ፡ ሰላም ፡ ጽናሕ ፡ ንስቲተ ፡ ወትሬኢ ፡ ወይትከሠት ፡ ለከ ፡ ኵሉ ፡ ዘኅቡእ ፡ ዘተከለ ፡ እግዚአ ፡ መናፍስት ። ወእሎንቱ ፡ አድባር ፡ ዘርኢከ ፡ ደብረ ፡ ሐፂን ፤ ወደብረ ፡ ፀሪቅ ፤ ወደብረ ፡ ብሩር ፤ ወደብረ ፡ ወርቅ ፤ ወደብረ ፡ ነጠብጣብ ፤ ወደብረ ፡ ዐረር ፤ እሉ ፡ ኵሎሙ ፡ ቅድሜሁ ፡ ለኅሩይ ፡ ይከውኑ ፡ ከመ ፡ መዓረ ፡ ግራ ፡ ዘቅድመ ፡ ገጸ ፡ እሳት ፡ ወከመ ፡ ማይ ፡ ዘይወርድ ፡ እምላዕሉ ፡ ዲበ ፡ እማንቱ ፡ አድባር ፡ ወይከውኑ ፡ ድኩማነ ፡ በቅድመ ፡ እገሪሁ ። ወይከውን ፡ በእማንቱ ፡ መዋዕል ፡ ኢይድኅኑ ፡ ኢበወርቅ ፡ ወኢበብሩር ፡ ወኢይክሉ ፡ ድኂነ ፡ ወጐይየ ። ወኢይከውን ፡ ሐፂን ፡ ለፀብእ ፡ ወኢልብስ ፡ ለድርዐ ፡ እንግድዓ ፡ እይበቍዕ ፡ ብርት ፡ ወኢናዕክ ፡ ኢይበቍዕ ፡ ወኢይትኌለቍ ፡ ወዐረር ፡ እይትፈቀድ ። እሉ ፡ ኵሎሙ ፡ ይትከሐዱ ፡ ወይትኀጐሉ ፡ ሀለዉ ፡ እምገጸ ፡ ምድር ፡ ሶበ ፡ ያስተርኢ ፡ ኅሩይ ፡ በቅድመ ፡ ገጹ ፡ ለአግዚአ ፡ መናፍስት ።} & \\
\textamh{{\Large 53}\ \  ወበህየ ፡ ርእያ ፡ አዕይንትየ ፡ ቈላ ፡ ዕሙቀ ፡ ወርኅው ፡ አፉሁ ፡ ወኵሎሙ ፡ እለ ፡ የኀድሩ ፡ ዲበ ፡ የብስ ፡ ወባሕር ፡ ወደሰያት ፡ አምኃ ፡ ወአስትዓ ፡ ወጋዳ ፡ ያመጽኡ ፡ ሎቱ ፡ ወዝክቱሰ ፡ ቈላ ፡ ዕሙቅ ፡ ኢይመልእ ። ወጌጋየ ፡ እደዊሆሙ ፡ ይገብሩ ፡ ወኵሎ ፡ ዘይጻምዉ ፡ ለጌጋይ ፡ ኃጥአን ፡ ይበልዑ ፡ ወእምገጹ ፡ ለእግዚአ ፡ መናፍስት ፡ ይትኀጐሉ ፡ ኃጥአን ፡ ወእምገጻ ፡ ለምድረ ፡ ዚአሁ ፡ ይትቀወሙ ፡ ወኢየኀልቁ ፡ ለዓለመ ፡ ዓለም ። እስመ ፡ ርኢክዎሙ ፡ ለመላእክተ ፡ መቅሠፍት ፡ እንዘ ፡ የሐውሩ ፡ ወያስተዳልዉ ፡ ኵሎ ፡ መባዕላተ ፡ ለሰይጣን ። ወተስእልክዎ ፡ ለመአከ ፡ ሰላም ፡ ዘየሐውር ፡ ምስሌየ ፡ እሎንተ ፡ መባዕላተ ፡ ለመኑ ፡ ያስተዳልውዎሙ ። ወይቤለኒ ፡ እሎንተ ፡ ያስተዳልውዎሙ ፡ ለነገሥት ፡ ወለኀያላን ፡ ዘዝንቱ ፡ ምድር ፡ ከመ ፡ ቦቱ ፡ ይትሐጐሉ ። ወእምድኅረ ፡ ዝንቱ ፡ ያስተርኢ ፡ ጻድቅ ፡ ወኅሩይ ፡ ቤተ ፡ ምስትጉቡአ ፡ ዚአሁ ፡ እምይእዜ ፡ ኢይትከልኡ ፡ በስሙ ፡ ለእግዚአ ፡ መናፍስት ። ወእሉ ፡ አድባር ፡ ይከውኑ ፡ በቅድመ ፡ ገጹ ፡ ከመ ፡ ምድር ፡ ወአውግር ፡ ይከውኑ ፡ ከመ ፡ ነቅዐ ፡ ማይ ፡ ወያዐርፉ ፡ ጻድቃን ፡ እም ፡ ጻማ ፡ ኃጥኣን ።} & \\
\textamh{{\Large 54}\ \  ወነጸርኩ ፡ ወተመየጥኩ ፡ ካልአ ፡ ገጸ ፡ ምድር ፡ ወርኢኩ ፡ ህየ ፡ ቈላ ፡ ዕሙቀ ፡ እንዘ ፡ ትነድድ ፡ እሳት ። ወአምጽእዎሙ ፡ ለነገሥት ፡ ወለኀያላን ፡ ወወደይዎሙ ፡ ውስተ ፡ ዕሙቅ ፡ ቈላ ። ወበህየ ፡ ርእያ ፡ አዕይንትየ ፡ ዘመባዕላቲሆሙ ፡ እንዘ ፡ ይገብርዎሙ ፡ መዓሠርተ ፡ ኀፂን ፡ ዘአልቦ ፡ መድሎት ። ወተስእልክዎ ፡ ለመልአከ ፡ ሰላም ፡ ዘየሐውር ፡ ምስሌየ ፡ እንዘ ፡ እብ ፡ እሉ ፡ ማእሥራተ ፡ መባዕላት ፡ ለመኑ ፡ ይዴለዉ ። ወይቤለኒ ፡ እሉ ፡ ይዴለዉ ፡ ለትዕይንተ ፡ አዛዝኤል ፡ ከመ ፡ ይትመጠውዎሙ ፡ ወይደይዎሙ ፡ መትሕተ ፡ ኵሉ ፡ ደይን ፡ ወአእባነ ፡ ጠዋያተ ፡ ይከድኑ ፡ መላትሒሆሙ ፡ በከመ ፡ አዘዘ ፡ እግዚአ ፡ መናፍስት ። ሚካኤል ፡ ወገብርኤል ፡ ሩፋኤል ፡ ወፋኑኤል ፡ ውእቶሙ ፡ ያጸንዕዎሙ ፡ በይእቲ ፡ ዕለት ፡ ዐባይ ፡ በውስተ ፡ ዕቶነ ፡ እሳት ፡ ዘይነድድ ፡ ይወድይዎሙ ፡ ውእተ ፡ ዕለተ ፡ ከመ ፡ ይትበቀል ፡ እምኔሆሙ ፡ እግዚአ ፡ መናፍስት ፡ በዐመፃሆሙ ፡ በእንተ ፡ ዘኮኑ ፡ ላእካነ ፡ ለሰይጣን ፡ ወአስሐትዎሙ ፡ ለእለ ፡ የኀድሩ ፡ ዲበ ፡ የብስ ። ወበውእቱ ፡ መዋዕል ፡ ይወጽእ ፡ መቅሠፍቱ ፡ ለእግዚአ ፡ መናፍስት ፡ ወይትረኀዉ ፡ ኵሉ ፡ መዛግብተ ፡ ማያት ፡ ዘመልዕልተ ፡ ሰማያት ፡ ወዲበ ፡ አንቅዕት ፡ እለ ፡ መትሕተ ፡ ሰማያት ፡ ወዘመትሕተ ፡ ምድር ። ወይዴመሩ ፡ ኵሎሙ ፡ ማያት ፡ ምስለ ፡ ማያት ፡ ዘመልዕልተ ፡ ሰማያት ፡ ማይሰ ፡ ዘመልዕልተ ፡ ሰማይ ፡ ተባዕታይ ፡ ውእቱ ፡ ወማይ ፡ ዘመትሕተ ፡ ምድር ፡ አንስታይ ፡ ይእቲ ። ወይደመሰሱ ፡ ኵሉ ፡ እለ ፡ የኀድሩ ፡ ዲበ ፡ የብስ ፡ ወእለ ፡ የኀድሩ ፡ መትሕተ ፡ አጽናፈ ፡ ሰማይ ። ወበእንተዝ ፡ አእመርዋ ፡ ለዐመፃሆሙ ፡ እንተ ፡ ገብሩ ፡ በዲበ ፡ ምድር ፡ ወበእንተዝ ፡ ይትኀጐሉ ።} & \\
\textamh{{\Large 55}\ \  ወእምድኅረዝ ፡ ነስሐ ፡ ርእሰ ፡ መዋዕል ፡ ወይቤ ፡ በከ ፡ አሕጐልክዎሙ ፡ ለኵሎሙ ፡ እለ ፡ ይነብሩ ፡ ዲበ ፡ የብስ ። ወመሐለ ፡ በስሙ ፡ ዐቢይ ፡ ከመ ፡ እምይእዜ ፡ ኢይገብር ፡ ከመዝ ፡ ለኵሎሙ ፡ እለ ፡ ይነብሩ ፡ ዲበ ፡ የብስ ፡ ወትእምርተ ፡ እወዲ ፡ በሰማያት ፡ ወይከውን ፡ ማእከሌየ ፡ ወማእከሎሙ ፡ ሃይማኖተ ፡ እስከ ፡ ለዓለም ፡ መጠነ ፡ መዋዕለ ፡ ሰማይ ፡ ዲበ ፡ ምድር ። ወእምዝ ፡ በትእዛዝየ ፡ ውእቱ ፡ ሶበ ፡ ፈቀድኩ ፡ ከመ ፡ አጽንዖሙ ፡ በእደ ፡ መላእክት ፡ በዕለተ ፡ ምንዳቤ ፡ ወሕማም ፡ ቅድመዝ ፡ መዓትየ ፡ ወመቅሠፍትየ ፡ የኀድር ፡ ላዕሌሆሙ ፡ መዓትየ ፡ ወመቅሠፍትየ ፡ ይቤ ፡ እግዚአብሔር ፡ እግዚአ ፡ መናፍስት ። ነገሥት ፡ ኀያላን ፡ እለ ፡ ተኀድሩ ፡ ውስተ ፡ የብስ ፡ ሀለወክሙ ፡ ትርአይዎ ፡ ለኅሩየ ፡ ዚአየ ፡ ከመ ፡ ይነብር ፡ በመንበረ ፡ ስብሐትየ ፡ ወይኴንኖ ፡ ለአዛዝኤል ፡ ወለኵሎሙ ፡ ማኅበረ ፡ ዚአሁ ፡ ወለትዕይንተ ፡ ዚአሁ ፡ ኵሎሙ ፡ በስሙ ፡ ለእግዚአ ፡ መናፍስት ።} & \\
\textamh{{\Large 108}\ \ ካልእ ፡ መጽሐፍ ፡ ዘጸሐፈ ፡ ሄኖክ ፡ ለወልዱ ፡ ማቱሳላ ፡ ወለእለ ፡ ይመጽኡ ፡ እምድኅሬሁ ፡ ወየዐቅብ ፡ ሥርዐተ ፡ በደኃሪ ፡ መዋዕል ። እለ ፡ ገበርክሙ ፡ ወትጸንሑ ፡ በእሉ ፡ መዋዕል ፡ እስከ ፡ ይትፌጸሙ ፡ እለ ፡ ይገብሩ ፡ እኩየ ፡ ወይትፌጸም ፡ ኀይሎሙ ፡ ለመአብሳን ፤ አንትሙሰ ፡ ጽንሑ ፡ እስክ ፡ ተሐልፍ ፡ ኀጢአት ፡ እስመ ፡ ሀሎ ፡ ስሞሙ ፡ ይደመሰስ ፡ እመጻሕፍተ ፡ ቅዱሳን ፡ ወዘርኦሙ ፡ ይትሐጐል ፡ ለዓለም ፡ ወመናፍስቲሆሙ ፡ ይትቀተሉ ፡ ወይጸርሑ ፡ ወየዐወይዉ ፡ በመካነ ፡ በድው ፡ ዘኢያስተርኢ ፡ ወበእሳት ፡ ይነድዱ ፡ እስመ ፡ ኢሀሎ ፡ ህየ ፡ ምድር ። ወርኢኩ ፡ ህየ ፡ ከመ ፡ ደመና ፡ ዘኢይትረአይ ፡ እስመ ፡ እምዕመቁ ፡ እክህልኩ ፡ ላዕለ ፡ ነጽሮ ፡ ወላህበ ፡ እሳቱ ፡ ርኢኩ ፡ እንዘ ፡ ይነድድ ፡ ስቡሕ ፡ ወይትከበቡ ፡ ከመ ፡ አድባር ፡ ስብሓን ፡ ወይትሀወኩ ፡ ለፌ ፡ ወለፌ ። ወተስእልክዎ ፡ ለ፩እመላእክት ፡ ቅዱሳን ፡ እለ ፡ ምስሌየ ፡ ወእቤሎ ፡ ምንት ፡ ውእቱ ፡ ዝስብሕ ፡ እስመ ፡ ኢኮነ ፡ ሰማየ ፡ አላ ፡ ላህበ ፡ እሳት ፡ ባሕቲቱ ፡ ዘይነድድ ፡ ወቃለ ፡ ጽራሕ ፡ ወብካይ ፡ ወአውያት ፡ ወሕማም ፡ ኀያል ። ወይቤለኒ ፡ ዝንቱ ፡ መካን ፡ ዘትሬኢ ፡ በህየ ፡ ይትወዶዩ ፡ መናፍስተ ፡ ኃጥኣን ፡ ወፅሩፋን ፡ ወእለ ፡ ይገብሩ ፡ እኩየ ፡ ወእለ ፡ ይመይጡ ፡ ኵሎ ፡ ዘነገረ ፡ እግዚአብሔር ፡ በአፈ ፡ ነቢያት ፡ እለ ፡ ሀለዉ ፡ ይትገበሩ ። እስመ ፡ ሀለዉ ፡ እምኔሆሙ ፡ ጽሑፋን ፡ ወልኩዓን ፡ ላዕለ ፡ በሰማይ ፡ ከመ ፡ ያንብብዎሙ ፡ መላእክት ፡ ወያእምሩ ፡ ዘሀሎ ፡ ይብጽሖሙ ፡ ለኃጥኣን ፡ ወለመናፍስተ ፡ ትሑታን ፡ ወእለ ፡ አሕመሙ ፡ ሥጋሆሙ ፡ ወተፈድዩ ፡ እምኀበ ፡ አምላክ ፡ ወእለ ፡ ኀስሩ ፡ እምእኩያን ፡ ሰብእ ፤ እለ ፡ አፍቀርዎ ፡ ለአምላክ ፡ ኢወርቀ ፡ ወኢብሩረ ፡ ኢያፍቀሩ ፡ ወኢኵሎ ፡ ሠናየ ፡ ዘውስተ ፡ ዓለም ፡ አላ ፡ ወሀቡ ፡ ሥጋሆሙ ፡ ለጻዕር ፤ ወእለ ፡ እምአመ ፡ ኮኑ ፡ ኢፈተዉ ፡ መባልዕተ ፡ ዘውስተ ፡ ምድር ፡ አላ ፡ ረሰዩ ፡ ርእሶሙ ፡ ከመ ፡ መንፈስ ፡ እንተ ፡ ኀለፈት ፡ ወዘንተ ፡ ዐቀቡ ፡ ወብዙኃ ፡ አመከሮሙ ፡ እግዚእ ፡ ወተረክብ ፡ መንፈሳቲሆሙ ፡ በንጽሕ ፡ ከመ ፡ ይባርክዎ ፡ ለስሙ ። ወኵሎ ፡ በረከቶሙ ፡ ነገርኩ ፡ በመጻሕፍት ፡ ወዐሰዮሙ ፡ ለአርእስቲሆሙ ፡ እስመ ፡ እሉ ፡ ተረክብ ፡ ያፈቅርዎ ፡ ለሰማይ ፡ እምእስትንፋሶሙ ፡ ዘለዓለም ፡ ወእንዘ ፡ ይትከየዱ ፡ እምእኩያን ፡ ሰብእ ፡ ወሰምዑ ፡ እምኀቤሆሙ ፡ ትዕይርተ ፡ ወጽርፈተ ፡ ወኀስሩ ፡ እንዘ ፡ ይባርኩኒ ። ወይእዜኒ ፡ እጼውዕ ፡ መናፍስቲሆሙ ፡ ለኄራን ፡ እምትውልድ ፡ እንተ ፡ ብርሃን ፡ ወእዌልጥ ፡ ለእለ ፡ ተወልዱ ፡ በጽልመት ፡ እለ ፡ በሥጋሆሙ ፡ እተፈድዩ ፡ ክብረ ፡ በከመ ፡ ይደሉ ፡ ለሃይማኖቶሙ ። ወአወጽኦሙ ፡ በብሩህ ፡ ብርሃን ፡ ለእለ ፡ ያፈቅርዎ ፡ ለስምየ ፡ ቅዱስ ፡ ወአነብር ፡ ፲፩ውስተ ፡ መንበረ ፡ ክብር ፡ ክብረ ፡ ዚአሁ ። ወይትወኀውኁ ፡ በአዝማን ፡ ዘአልቦ ፡ ኍልቍ ፡ እስመ ፡ ጽድቅ ፡ ኵነኔሁ ፡ ለአምላክ ፡ እስመ ፡ ለመሃይምናን ፡ ሃይማኖተ ፡ ይሁብ ፡ በማኅደረ ፡ ፍናዋት ፡ ርቱዓት ። ወይሬእይዎሙ ፡ ለእለ ፡ ተወልዱ ፡ በጽልመት ፡ ይትወደዩ ፡ በጽልመት ፡ እንዘ ፡ ይትወኀውኁ ፡ ጻድቃን ። ወይጸርሑ ፡ ወይሬእይዎሙ ፡ ኃጥኣን ፡ እንዘ ፡ ይበርሁ ፡ ወየሐውሩ ፡ እሙንቱሂ ፡ በኀበ ፡ ተጽሕፈ ፡ ሎሙ ፡ መዋዕል ፡ ወአዝማን ።} & \\
\end{longtable}
\end{document}
